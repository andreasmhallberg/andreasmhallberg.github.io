\documentclass{article}
\usepackage{tikz}
\usepackage{calc}
\usepackage{kantlipsum}

% lengths for tikz plots
\newlength\plotheight % Height of plotting area
\setlength\plotheight{.2\textheight}
\newlength\plotwidth % Width of plotting area
\setlength\plotwidth{.7\textwidth}
\newlength\axissep % Space between plotting area and axis
\setlength\axissep{1em}
\newlength\tickl % Length of minor ticks
\setlength\tickl{2mm}
\newlength\ltickl % Length of major ticks
\setlength\ltickl{3mm} 
\newlength\ylabsep % space between plotting area and y-label 
\setlength\ylabsep{\axissep+\tickl+1.5em}
\newlength\xlabsep % space between plotting area and x-label
\setlength\xlabsep{\axissep+\tickl+1.5em}

\begin{document}

\kant[1]

\begin{figure}
\center
\def\maxy{100}%
\def\miny{0}%
\def\maxx{3}%
\def\minx{0}%
\def\xlab{Definiteness}%
\def\ylab{Case marking (\%)}%
\begin{tikzpicture}[
    y=\plotheight/(\maxy-\miny)
    , x=\plotwidth/(\maxx-\minx)]

    \useasboundingbox 
        (-\axissep,-\axissep-\tickl-\xlabsep)
        rectangle  (\maxx,\maxy);

% y-axis
\draw (\minx-\axissep,\miny) -- (\minx-\axissep,\maxy);
% y-ticks
\foreach \x in {0,25,...,\maxy} 
    {\draw (\minx,\x) ++ (-\axissep,0) -- ++ (-\tickl,0)
% y-ticklabels    
	    node[anchor=east] {\x};}
% y-label
    \path  (\minx-\ylabsep, {(\miny+\maxy)/2}) node[align=center, rotate=90 ,anchor=south] {\ylab};

% x-axis
\draw (\minx,\miny) ++ (0,-\axissep) -- ++ (\maxx,0);

% x-ticks
\foreach \x/\y in { 0, 1, 2, 3 }
    \draw (\x, \miny) ++ (0,-\axissep) -- ++ (0, -\tickl);
% x-ticklabels
      \node at (0,-\axissep-\tickl) [align=center,anchor=north ] {Definite\\article};
      \node at (1,-\axissep-\tickl) [anchor=north] {\textsc{cs-n/c}\vphantom{I}}; % \vphantom to align
      \node at (2,-\axissep-\tickl) [anchor=north] {Indefinite};
      \node at (3,-\axissep-\tickl) [align=center,anchor=north ] {Enclitic\\pronoun};

% x-label
    \path ({(\minx+\maxx)/2},\miny) ++ (0, -\xlabsep-\baselineskip)
    node[anchor=north] {\xlab};

% \newcommand\plotdata[4]{
%   \draw[fill=black] (0,#1) circle [radius=1pt];
%   \draw[fill=black] (1,#2) circle [radius=1pt];
%   \draw[fill=black] (2,#3) circle [radius=1pt];
%   \draw[fill=black] (3,#4) circle [radius=1pt];
% }

\newcommand\plotdata[4]{
  \draw (0,#1) -- (1,#2) -- (2,#3) -- (3,#4);
}

\plotdata{10.800}{70.000}{70.149}{99.999}
\plotdata{2.967}{15.277}{24.902}{97.826}
\plotdata{0.809}{19.718}{30.496}{93.750}
\plotdata{4.072}{12.121}{12.413}{56.249}
\plotdata{0.684}{4.081}{13.872}{47.999}

\end{tikzpicture}
\caption{Predicted case marking by types of definiteness}
\end{figure}

\kant[2]

\end{document}
