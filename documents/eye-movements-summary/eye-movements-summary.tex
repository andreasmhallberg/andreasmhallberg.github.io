\documentclass{article}
\usepackage[dvipsnames]{xcolor}
\usepackage{\string ~/mylatexstuff/ccbyandreas}
\usepackage[a4paper, landscape, top=2cm, bottom=2cm, left=2cm, right=2cm]{geometry}
\usepackage{etoolbox}
\usepackage{graphicx}
\usepackage{footnote} % to save and spew notes
\usepackage{ccicons}
\usepackage{calc}
\def\UrlFont{\rmfamily\itshape} % roman font in urls
\usepackage{ragged2e}
\usepackage{tikz}
\usepackage{varwidth} % nodes size with adjusted width
\usetikzlibrary{calc,backgrounds}
\usepackage{polyglossia}
\setmainlanguage{english}
\setotherlanguage{arabic}

\usepackage{fontspec}
\setmainfont[Numbers=OldStyle]{Linux Libertine O}
\setsansfont[Numbers=OldStyle]{Source Sans Pro ExtraLight}


\def\fixcolor{blue}
\def\saccolor{red}
\def\spacolor{green}


% Lengths
\newlength\infosep % Between example line and infoboxes
\setlength{\infosep}{2cm}


% tikzstyles
\tikzstyle{infoboxfix}=[
    color=\fixcolor
  , font=\itshape\sffamily\small
]
\tikzstyle{infoboxsac}=[
    color=\saccolor
  , font=\itshape\sffamily\small
]
\tikzstyle{fixline}=[
    color=\fixcolor
  , shorten < = 4pt
  , shorten > = 15pt
]
\tikzstyle{sacline}=[
    color=\saccolor
  , shorten < = 4pt
  , shorten > = 15pt
]


% Drawing commands
\newcommand{\fix}[2]{%
    \node [inner sep = 0pt](#2) {#1};
    \path[draw=\fixcolor, fill=\fixcolor]  (#2.south) ++ (0,.5ex) circle (2pt);%
}

% ARguments are coordinates specified with \tikzmark
\newcommand{\saccade}[2]{%
    \draw[->, color=\saccolor, shorten < =1pt, shorten > =1pt] (#1) -- (#2);
}

\newcommand{\regression}[2]{%
  \draw[->, overlay, draw=\saccolor, shorten < =1pt, shorten > =1pt] (#1) -- (#2);
}

% \fixinfo{node name}{text}
\newcommand{\fixinfo}[3][0pt]{%
\draw [fixline] (#2) -- ++ (#1, \infosep) node [infoboxfix] {\begin{varwidth}{3cm}#3\end{varwidth}} ;
}

% \sacinfo[hshift]{node1}{node2}{text}
\newcommand{\sacinfo}[4][0pt]{%
  \draw[sacline] ($(#2)!.5!(#3)$) -- node [infoboxsac] {\begin{varwidth}{3cm}#4\end{varwidth}} ;
}

\begin{document}



\null\vspace{5cm}




% Infoboxes

\center

\begin{tikzpicture}[inner sep=0pt]

% The example sentence

  \node [color=black!20, font=\ttfamily\huge ] (example) {%
    This ex\fix{a}{fix1}mple sentence ill\fix{u}{fix2}stra\fix{t}{fixmult}es ty\fix{p}{fix3}ical eye mo\fix{v}{fix4}ement pa\fix{t}{fix5}terns.%
  };


  % \fixinfo{sentensbegin}{New lines are fixated 5--7 letters in, so that around 80\% of the line are withing lateral fixations.}
  \fixinfo{fix2}{Words of 8 letters or more are almost always fixated.}
  \fixinfo{fix1}{Fixations are 200--250~ms. long.}
  % \fixinfo{shortword}{Words of 2--3 letters are fixated 25\% of the time.} ;
  \fixinfo{fix5}{Long words are often refixated.}

  % \sacinfo{fix1}{fix2}{Saccades are typically \mbox{6--12} letters and take 20--30ms.}

  \saccade{fix1}{fix2}
  \saccade{fix2}{fix3}
  \saccade{fix3}{fix4}
  \saccade{fix4}{fix5}


\end{tikzpicture}

\vfill
\small
\begin{verbatim}
- in other orthografies
- fixation times
- saccade length
- perceptual span
\end{verbatim}


\end{document}
