\documentclass{article}
\usepackage[dvipsnames]{xcolor}
\usepackage{\string ~/mylatexstuff/ccbyandreas}
\usepackage[landscape, top=2cm, bottom=2cm, left=2cm, right=2cm]{geometry}
\usepackage{etoolbox}
\usepackage{graphicx}
\usepackage{footnote} % to save and spew notes
\usepackage{ccicons}
\usepackage{calc}
\def\UrlFont{\rmfamily\itshape} % roman font in urls
\usepackage{ragged2e}
\usepackage{tikz}
\usepackage{varwidth} % nodes size with adjusted width
\usetikzlibrary{calc,backgrounds,tikzmark}
\usepackage{polyglossia}
\setmainlanguage{english}
\setotherlanguage{arabic}

\usepackage{fontspec}
\setmainfont[Numbers=OldStyle]{Linux Libertine O}
\setsansfont[Numbers=OldStyle]{Source Sans Pro ExtraLight}


\def\fixcolor{blue}
\def\saccolor{red}
\def\spacolor{green}
\def\spacolor{green}


\newcommand{\fix}[2]{%
  \tikzmark{#2a}%
  \begin{tikzpicture}
    \node [inner sep = 0pt, text depth=0pt](fix) {#1};
    \draw[draw=\fixcolor, fill=\fixcolor] ($ (fix.south west)!0.5!(fix.south east) $) ++ (0,.5ex) circle (2pt);
  \end{tikzpicture}%
  \tikzmark{#2b}%
}

% ARguments are coordinates specified with \tikzmark
\newcommand{\saccade}[2]{%
  \begin{tikzpicture}[remember picture]
    \draw[->, overlay, draw=\saccolor, shorten < =1pt, shorten > =1pt] ($(pic cs:#1) + (0,.5ex)$) -- ($(pic cs:#2) + (0,.5ex)$);
  \end{tikzpicture}%
}

\newcommand{\regression}[2]{%
\begin{tikzpicture}[remember picture, above=2ex]
  \draw[->, overlay, draw=\saccolor, shorten < =1pt, shorten > =1pt] ($(pic cs:#1) + (0,.5ex)$) -- ($(pic cs:#2) + (0,.5ex)$);
\end{tikzpicture}%
}


\tikzstyle{infoboxfix}=[
    color=\fixcolor
  , font=\itshape\sffamily\small
  , yshift=\infosep
]
\tikzstyle{infoboxsac}=[
    color=\saccolor
  , font=\itshape\sffamily\small
  , above=1.5\infosep
]
\tikzstyle{fixline}=[
    color=\fixcolor
  , shorten < = 4pt
  , shorten > = 15pt
]
\tikzstyle{sacline}=[
    color=\saccolor
  , shorten < = 4pt
  , shorten > = 15pt
]

% \fixinfo{tikzmark name}{text}
\newcommand{\fixinfo}[2]{%
  \coordinate (#1) at ($(pic cs:#1a)!.5!(pic cs:#1b)$);
  \node[infoboxfix] (infobox) at (#1) {\begin{varwidth}{3cm}#2\end{varwidth}} ;
  \draw[fixline] (infobox) -- (#1);
}

% \fixinfo{tikzmark name}{text}
% Use to give info on fixations when the position is set with \tikzmark and there is no fixation point
\newcommand{\fixinfonofix}[2]{%
\node[infoboxfix] (infobox) at (pic cs:#1) {\begin{varwidth}{3cm}#2\end{varwidth}} ;
  \draw[fixline] (infobox) -- (pic cs:#1);
}

% \sacinfo{tikzmark 1}{tikzmark2}{text}
\newcommand{\sacinfo}[3]{%
  \coordinate (sac) at ($(pic cs:#1b)!.5!(pic cs:#2a)$);
\node[infoboxsac] (infobox) at (sac) {\begin{varwidth}{3cm}#3\end{varwidth}} ;
  \draw[sacline] (infobox) -- (sac);
}


% Lengths
\newlength\infosep
\setlength{\infosep}{2cm}
\newlength\letterwidth
\setlength{\letterwidth}{\widthof{\huge h}}

\begin{document}

\begin{tikzpicture}[remember picture, overlay]
  \draw [color=green] (pic cs:fix1a) ++ (6\letterwidth,.5ex) ellipse (\letterwidth*9 and 10mm);
  % \draw [color=green!20] (pic cs:fix2a) ++ (6\letterwidth,.5ex) ellipse (\letterwidth*9 and 10mm);
  % \draw [color=green!20] (pic cs:fix3a) ++ (6\letterwidth,.5ex) ellipse (\letterwidth*9 and 10mm);
  % \draw [color=green!20] (pic cs:fix4a) ++ (6\letterwidth,.5ex) ellipse (\letterwidth*9 and 10mm);
  % \draw [color=green!20] (pic cs:fix5a) ++ (6\letterwidth,.5ex) ellipse (\letterwidth*9 and 10mm);
  % \draw [color=green!20] (pic cs:fix6a) ++ (6\letterwidth,.5ex) ellipse (\letterwidth*9 and 10mm);
\end{tikzpicture}

When reading, the eyes do not move in smooth motion but in a combination of extremely quick motions, called \textcolor{\saccolor}{saccades} and stops, called \textcolor{\fixcolor}{fixations}.

\ttfamily\huge



\null\vspace{5cm}

% The example sentence


\textcolor{black!20}{%
  \tikzmark{sentensbegin}%
  This ex\fix{a}{fix1}mple ill\fix{u}{fix2}stra\fix{t}{fix3}es e\tikzmark{shortword}ye mo\fix{v}{fix4}ements ty\fix{p}{fix5}ical o\tikzmark{function}f re\fix{a}{fix6}ding.%
  \tikzmark{sentensend}%
}

\saccade{fix1b}{fix2a}
\saccade{fix2b}{fix3a}
\saccade{fix3b}{fix4a}
\saccade{fix4b}{fix5a}
\saccade{fix5b}{fix6a}



% Infoboxes

\begin{tikzpicture}[remember picture, overlay, inner sep=0pt]

  \fixinfonofix{sentensbegin}{New lines are fixated 5--7 letters in, so that around 80\% of the line are withing lateral fixations.}

  \fixinfo{fix2}{Words of 8 letters or more are almost always fixated.}

  \fixinfo{fix1}{Fixations are 200--250~ms. long.}

  \fixinfonofix{shortword}{Words of 2--3 letters are fixated 25\% of the time.} ;
  \fixinfonofix{shortword}{Function words are fixated 35\% because they short and often predictable from contexts.} ;

  \fixinfo{fix5}{Long words are often refixated.}

  \sacinfo{fix1}{fix2}{Saccades are typically \mbox{6--12} letters and take 20--30ms.}


\end{tikzpicture}

\vfill
\small
\begin{verbatim}
- in other orthografies
- fixation times
- saccade length
- perceptual span
\end{verbatim}


\end{document}
