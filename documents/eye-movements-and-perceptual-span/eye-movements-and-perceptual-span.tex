\documentclass{article}
\usepackage[dvipsnames]{xcolor}
\usepackage{\string ~/mylatexstuff/ccbyandreas}
\usepackage[
   landscape
, top=2cm
, bottom=2cm
, hmargin=3.5cm
, a4paper
]{geometry}
\usepackage{etoolbox}
\usepackage{ifthen}
\usepackage{graphicx}
\usepackage{multicol}
\usepackage{microtype}
\usepackage{footnote} % to save and spew notes
\usepackage{ccicons}
\usepackage{calc}
\def\UrlFont{\rmfamily\itshape} % roman font in urls
\usepackage{ragged2e}
\usepackage{tikz}
\usepackage{varwidth} % nodes size with adjusted width
\usetikzlibrary{arrows,fadings,calc,backgrounds,tikzmark}
\usepackage{polyglossia}
\setmainlanguage{english}
\setotherlanguage{arabic}

\usepackage{fontspec}
\setmainfont[Numbers=OldStyle]{Linux Libertine O}
\setsansfont[Numbers=OldStyle]{Source Sans Pro Light}

\frenchspacing

\def\fixcolor{blue!30!gray}
\def\saccolor{red!50!gray}

% Lengths
\newlength\infosep
\setlength{\infosep}{2cm}
\newlength\infoseptop
\setlength{\infoseptop}{4.5cm}
\newlength\letterwidth
\setlength{\letterwidth}{\widthof{\huge h}}

\setlength\parindent{0pt}

\newcommand{\fix}[2]{%
  \tikzmark{#2a}%
  \begin{tikzpicture}
    \node [inner sep = 0pt, text depth=0pt](fix) {#1};
    \draw[fill opacity=.7, draw=\fixcolor, fill=\fixcolor]
      ($ (fix.south west)!0.5!(fix.south east) $)
      ++ (0,.5ex) circle (4pt);
  \end{tikzpicture}%
  \tikzmark{#2b}%
}

% ARguments are coordinates specified with \tikzmark
\newcommand{\saccade}[2]{%
  \begin{tikzpicture}[remember picture]
    \draw[->,>=latex, overlay, draw=\saccolor, shorten < =1pt, shorten > =1pt] ($(pic cs:#1) + (0,.5ex)$) to ($(pic cs:#2) + (0,.5ex)$);
  \end{tikzpicture}%
}


\newcommand{\regfix}[2]{%
  \tikzmark{#2a}%
  \begin{tikzpicture}
    \node [inner sep = 0pt, text depth=0pt](fix) {#1};
    \draw[fill opacity=.7, draw=\fixcolor, fill=\fixcolor, transform canvas={yshift=-1.7ex}]
      ($ (fix.south west)!0.5!(fix.south east) $)
      ++ (0,.5ex) circle (4pt);
  \end{tikzpicture}%
  \tikzmark{#2b}%
}

\newcommand{\regression}[2]{%
\begin{tikzpicture}[remember picture]
  \draw[->,>=latex, overlay, draw=\saccolor, shorten < =1pt, shorten > =1pt, transform canvas={yshift=-1.5ex}]
    ($(pic cs:#1) + (0,.5ex)$) --
    ($(pic cs:#2) + (0,.5ex)$);
\end{tikzpicture}%
}

\newcommand{\regressiontwo}[2]{%
\begin{tikzpicture}[remember picture]
  \draw[->,>=latex, overlay, draw=\saccolor, shorten < =1pt, shorten > =1pt, transform canvas={yshift=-2ex}]
    ($(pic cs:#1) + (0,.5ex)$) --
    ($(pic cs:#2) + (0,.5ex)$);
\end{tikzpicture}%
}



\tikzstyle{infoboxfix}=[
    color=\fixcolor
  , font=\itshape\sffamily\footnotesize
  % , yshift=\infosep
]
\tikzstyle{infoboxsac}=[
    color=\saccolor
  , font=\itshape\sffamily\footnotesize
  % , above=1.5\infosep
]
\tikzstyle{fixline}=[
    color=\fixcolor
  , shorten < = 4pt
  , shorten > = 15pt
  , ultra thin
]
\tikzstyle{sacline}=[
    color=\saccolor
  , shorten < = 4pt
  , shorten > = 15pt
  , ultra thin
]

% \fixinfo{tikzmark name}{text}
\newcommand{\fixinfo}[3][1]{%
  \coordinate (#2) at ($(pic cs:#2a)!.5!(pic cs:#2b)$);
  \node[infoboxfix, yshift=#1\infosep]
    (infobox) at (#2)
    {\begin{varwidth}{2.3cm}\Centering\color{\fixcolor}#3\end{varwidth}} ;
  \draw[fixline] (infobox) -- (#2);
}

% \fixinfo{tikzmark name}{text}
% Use to give info on fixations when the position is set with \tikzmark and there is no fixation point
\newcommand{\fixinfonofix}[3][1]{%
\node[infoboxfix, yshift=#1\infosep] (infobox) at (pic cs:#2) {\begin{varwidth}{2.3cm}\Centering\color{\fixcolor}#3\end{varwidth}} ;
  \draw[fixline] (infobox) -- (pic cs:#2);
}

% \sacinfo[multiplication to infosep]{tikzmark 1}{tikzmark2}{text}
\newcommand{\sacinfo}[4][1]{%
  \coordinate (sac) at ($(pic cs:#2b)!.5!(pic cs:#3a)$);
\node[infoboxsac, yshift=#1\infosep] (infobox) at (sac) {\begin{varwidth}{2.3cm}\Centering\color{\saccolor}#4\end{varwidth}} ;
  \draw[sacline] (infobox) -- (sac);
}



\pagestyle{empty}

\begin{document}


% title
\centerline{\Huge 
\scshape Eye Movements and Perceptual Span in Reading
}

\bigskip
\centerline{Based on Rayner, K., 1998. "Eye movements in reading and information processing."  \textit{Psychological Bulletin}, 85(3), pp.618--66.}

\bigskip


% The example sentence

\begin{multicols}{3}
  \setlength\parfillskip{0pt}
  \setlength\emergencystretch{40pt}
\itshape
When reading, the eyes do not move along a line of text in a smooth motion, but in a combination of extremely quick motions, called \textcolor{\saccolor}{\mbox{saccades}}, and stops, called \textcolor{\fixcolor}{\mbox{fixations}}. It is only during fixations that we retrieve information. The saccades are extremely fast (500°/s) so that any vision during them would be perceived as a blur. This is however filtered out by the brain so that during saccades we are effectively blind. The text is thus taken in as a sequence of windows around each fixation. This is called the perceptual span. The perceptual span is asymmetric. In English it reaches 14--15~letters to the right of the fixation and 4--5~letters to the left. Words are only identifiable 7--8 letters to right. Further to the right we primarily get information on word length which is used to plan the next saccade. For languages that are read from left to right the asymmetry of the perceptual span is mirrored. In languages with denser writing systems it is narrower, making saccades shorter (2--3 characters in Chinese, 5--6 letters in Hebrew), but fixation times remain the same. 
\end{multicols}


\vspace{3.5cm}


{%
\color{black!70}
\ttfamily\huge
  %
  Thi\tikzmark{sentensbegin}s ex\fix{a}{fix1}mple ill\fix{u}{fix2}stra\fix{t}{fix3}es e\tikzmark{shortword}ye mo\fix{v}{fix4}em\regfix{e}{regfix}nts ty\fix{p}{fix5}ical o\tikzmark{function}f rea\fix{d}{fix6}ing.%
  \tikzmark{sentensend}%

\saccade{fix1b}{fix2a}
\saccade{fix2b}{fix3a}
\saccade{fix3b}{fix4a}
\saccade{fix4b}{fix5a}
% \saccade{fix5b}{fix6a}
\regression{fix5a}{regfixb}
\regressiontwo{regfixb}{fix6a}

\vspace{-1.3cm}

% Infoboxes

\begin{tikzpicture}[remember picture, overlay, inner sep=0pt]

  \fixinfonofix{sentensbegin}{There is no fixation on the start of a line. The first fixation is typically 5--7 letters in.}

  \fixinfo[1.7]{fix2}{Words of eight or more letters are almost always fixated.}

  \fixinfo[1.7]{fix1}{Most fixations are 200--250~ms. long.}

  \fixinfonofix[1.7]{shortword}{Words of 2--3 letters are only fixated 25\% of the time.} ;

  \fixinfonofix[1.9]{function}{Function words are only fixated 35\% of the time.} ;

  \fixinfo{fix3}{Long words are often re-fixated.}

  \sacinfo{fix1}{fix2}{Saccades are \mbox{6--12} letters long and take 20--30ms.}
  % \sacinfo{fix4}{fix5}{Saccades or so fast (500°/s) that during them we are effectively blind.}

  \sacinfo{fix5}{regfix}{10-15\% of all saccades move backwards and are called re\-gressions. They most often target the previous word.}

  \fixinfo{fix6}{Fixation times are longer at the end of a clause or sentence as the proposition is processed.}

  \sacinfo{regfix}{fix6}{After regressions readers often skip parts they have already read, giving very long saccades (15--20~letters).}

\end{tikzpicture}


{\huge\ttfamily 
  This ex\fix{a}{span1}mple ill{u}stra{t}es eye mo{v}ements ty{p}ical of re{a}ding.%

  This example ill\fix{u}{span2}stra{t}es eye mo{v}ements ty{p}ical of re{a}ding.%

  This example ill{u}stra\fix{t}{span3}es eye mo{v}ements ty{p}ical of re{a}ding.%

  This example ill{u}stra{t}es eye mo\fix{v}{span4}ements ty{p}ical of re{a}ding.%

  This example ill{u}stra{t}es eye mo{v}ements ty\fix{p}{span5}ical of re{a}ding.%

  This example ill{u}stra{t}es eye mo{v}em\fix{e}{reg}nts typical of re{a}ding.%

  This example ill{u}stra{t}es eye mo{v}ements ty{p}ical of re\fix{a}{span6}ding.%

}%\huge ends

\tikzfading[name=fade right,
  left color=transparent!0,
  right color=transparent!100]
\tikzfading[name=fade left,
  left color=transparent!110,
  right color=transparent!0]


\newcommand\thespan[1]{%
\begin{tikzpicture}[remember picture, overlay, inner sep=0pt]
  \fill [path fading=fade right,  white](pic cs:#1a) ++ (0,-1ex) rectangle ++ (-5\letterwidth, 3ex); 
  \fill [fill=white](pic cs:#1a)  ++ (-5\letterwidth,-1ex) rectangle ++ (-\textwidth, 3ex); 
  \fill [path fading=fade left, white](pic cs:#1b) ++ (0,-1ex) rectangle ++ (14\letterwidth, 3ex); 
  \fill [fill=white](pic cs:#1b) ++ (14\letterwidth, -1ex) rectangle ++ (\textwidth,3ex); 
\end{tikzpicture}%
}

\thespan{span1}
\thespan{span2}
\thespan{span3}
\thespan{span4}
\thespan{span5}
\thespan{reg}
\thespan{span6}
}% \huge ends here

\vfill

\end{document}
