% !cp '%'.pdf ~/blog/andreasmhallberg.github.io/documents/arabic-letters.tex.pdf
% 171201

\documentclass[oneside]{article}
% \usepackage[all, placement=bottom, scale=1, color=black, opacity=1, vshift=1cm]{background}
\usepackage{\string ~/mylatexstuff/ccbyandreas}
\usepackage{array}
\usepackage[
  left=2.3cm
  , right=6cm
  , marginparwidth=4cm
  , vmargin={3cm}
% , showframe
]{geometry}
\usepackage[compact, sc]{titlesec}
\titlelabel{\thetitle.\ }
\usepackage{booktabs}
\usepackage{microtype}
\usepackage{bigdelim}
\usepackage{tabularx}
\usepackage{calc}
\usepackage{xcolor}
\usepackage{marginnote}
\usepackage{colortbl} % To highlight cells in tables. See \mc command.
% Mark single cells
\newcommand\mc{\cellcolor{black!10}}
\usepackage{polyglossia}
\setmainfont[Numbers=OldStyle]{Linux Libertine O}
\setmainlanguage{english}
\setotherlanguage{arabic}
\usepackage{multirow}

\renewcommand*{\marginfont}{\normalsize\itshape\sloppy}
\renewcommand*{\raggedleftmarginnote}{}
\renewcommand*{\raggedrightmarginnote}{}

\setlength\parindent{0pt}

% command to raise letters yaa in table for better viual palance and to avoid clashing
\newcommand{\raiseyaa}[1]{\raisebox{2pt}{#1}}

\newfontfamily\arabicfont[Script=Arabic,Scale=1.5]{Scheherazade}

\newcommand\ligaturetable{%
\renewcommand\arraystretch{1.2}%
    \upshape
    \framebox{%
\newfontfamily\arabicfont[Script=Arabic]{Amiri}%
\begin{tabularx}{\linewidth}{r@{~\color{gray}$\rightarrow$~}rr@{~~}>{\itshape}X@{}}
    \multicolumn{4}{l}{\textbf{Common ligatures}}\\
    \addlinespace
    \textarabic{لـ\char"200D ا}     & \textarabic{لا}\rlap{*\ }    & \textarabic{سلام}     & yallaa   \\  
    \textarabic{لـمـ}                  & \textarabic{لمـ}  & \textarabic{المسلم} & \mbox{al-muslim} \\ 
  \textarabic{فـي}                 & \textarabic{في}   &                     & fii \\           
\textarabic{يـحـ}                  & \textarabic{يحـ}  & \textarabic{يحيا}   & yaHyaa  \\   
\textarabic{تـجـ}                  & \textarabic{تجـ}  & \textarabic{تجارة}  & tijaara \\   
\textarabic{مـحـ}                  & \textarabic{محـ}  & \textarabic{محل}    & maHall\\     
\textarabic{مـمـ}                  & \textarabic{ممـ}  & \textarabic{ممل}    & mumill\\     
\textarabic{الـلـه}                & \textarabic{الله} &                     & allaah \\           
\addlinespace
\multicolumn{4}{l}{* Non-optional}\\



\end{tabularx}%
}}

\pagestyle{empty}

\frenchspacing

\begin{document}
\Large

% columnwidth
    \newlength\alphcolw
    \setlength\alphcolw{3em}

% baselinerule
    \newcommand{\br}[1]{%
    % \rlap{\hspace{-.3em}\color{red}\rule{.5\alphcolw}{.1pt}}%
        \rlap{\color{red}\hspace{-3pt}\rule{\widthof{#1}+8pt}{.2pt}}#1%
}


{\fontspec{Linux Libertine O}\huge\itshape The Arabic writing system}%
\marginnote{This chart was designed as a complement to the \textup{Alif Baa} Arabic textbook (George\-town University Press, 1995) and follows the transcription system used therein.}[-1cm]
\\{\small\itshape\today}

\vfill

\section{The Alphabet}

\bigskip

\begin{tabular}{>{\strut\itshape}l>{\itshape}cccccc} 
   % \toprule
   \multicolumn{1}{l}{%
   \makebox[\alphcolw][l]{Name}} & \multicolumn{1}{l}{\makebox[\alphcolw][c]{Trans.}}  & \makebox[\alphcolw][c]{Isolated}& \makebox[\alphcolw][c]{Final} & \makebox[\alphcolw][c]{Medial} & \makebox[\alphcolw][c]{Initial}   \\
\midrule

       \marginnote{Gray \colorbox{black!10}{background} marks letters that do not connect forward. A following letter takes the initial or isolated form.}%
alif         & aa   & \mc\textarabic{ا}   & \mc\textarabic{ـا}    & \mc\textarabic{ـا }   & \mc\textarabic{ا}    \\
baaʾ         & b    & \textarabic{ب}      & \textarabic{ـب}       & \textarabic{ـبـ}      & \textarabic{بـ}      \\
taaʾ         & t    & \textarabic{ت}      & \textarabic{ـت}       & \textarabic{ـتـ}      & \textarabic{تـ}      \\
thaaʾ        & th   & \textarabic{ث}      & \textarabic{ـث}       & \textarabic{ـثـ}      & \textarabic{ثـ}      \\
\marginnote{The baseline is marked with a \br{red line} on descending letter forms.}%
jiim          & j    & \br{\textarabic{ج}}   & \br{\textarabic{ـج}}    & \textarabic{ـجـ}      & \textarabic{جـ}      \\
Haaʾ         & H    & \br{\textarabic{ح}}   & \br{\textarabic{ـح}}    & \textarabic{ـحـ}      & \textarabic{حـ}      \\
khaaʾ        & kh   & \br{\textarabic{خ}}   & \br{\textarabic{ـخ}}    & \textarabic{ـخـ}      & \textarabic{خـ}      \\
daal         & d    & \mc\textarabic{د}   & \mc\textarabic{ـد}    & \mc\textarabic{ـد}    & \mc\textarabic{د}    \\
dhaal        & dh   & \mc\textarabic{ذ}   & \mc\textarabic{ـذ}    & \mc\textarabic{ـذ}    & \mc\textarabic{ذ}    \\
raaʾ         & r    & \mc\br{\textarabic{ر}}& \mc\br{\textarabic{ـر}} & \mc\br{\textarabic{ـر}} & \mc\br{\textarabic{ر}} \\
zaay         & z    & \mc\br{\textarabic{ز}}& \mc\br{\textarabic{ـز}} & \mc\br{\textarabic{ـز}} & \mc\br{\textarabic{ز}} \\
siin         & s    & \br{\textarabic{س}}   & \br{\textarabic{ـس}}    & \textarabic{ـسـ}      & \textarabic{سـ}      \\
shiin        & sh   & \br{\textarabic{ش}}   & \br{\textarabic{ـش}}    & \textarabic{ـشـ}      & \textarabic{شـ}      \\
\marginnote{\textarabic{ص}, \textarabic{ض}, \textarabic{ط} and \textarabic{ظ} are so called emphatic letters and affect quality of surrounding vowels.}[-1ex]%
Saad         & S    & \br{\textarabic{ص}}   & \br{\textarabic{ـص}}    & \textarabic{ـصـ}      & \textarabic{صـ} \rlap{\color{gray}\hspace{1em}\ldelim]{4}{3mm}}     \\
Daad         & D    & \br{\textarabic{ض}}   & \br{\textarabic{ـض}}    & \textarabic{ـضـ}      & \textarabic{ضـ}      \\
Taaʾ         & T    & \textarabic{ط}      & \textarabic{ـط}       & \textarabic{ـطـ}      & \textarabic{طـ}      \\
\marginnote{\small\ligaturetable}%
 DHaaʾ       & DH   & \textarabic{ظ}      & \textarabic{ـظ}       & \textarabic{ـظـ}      & \textarabic{ظـ}      \\
 \llap{ʿ}ayn & ʿ    & \br{\textarabic{ع}}   & \br{\textarabic{ـع}}    & \textarabic{ـعـ}      & \textarabic{عـ}      \\
ghayn        & gh   & \br{\textarabic{غ}}   & \br{\textarabic{ـغ}}    & \textarabic{ـغـ}      & \textarabic{غـ}      \\
faaʾ         & f    & \textarabic{ف}      & \textarabic{ـف}       & \textarabic{ـفـ}      & \textarabic{فـ}      \\
qaaf         & q    & \br{\textarabic{ق}}   & \br{\textarabic{ـق}}    & \textarabic{ـقـ}      & \textarabic{قـ}      \\
kaaf         & k    & \textarabic{ك}      & \textarabic{ـك}       & \textarabic{ـكـ}      & \textarabic{كـ}      \\
laam         & l    & \br{\textarabic{ل}}      & \br{\textarabic{ـل}}    & \textarabic{ـلـ}      & \textarabic{لـ}      \\
miim         & m    & \br{\textarabic{م}}   & \br{\textarabic{ـم}}    & \textarabic{ـمـ}      & \textarabic{مـ}      \\
nuun         & n    & \br{\textarabic{ن}}   & \br{\textarabic{ـن}}    & \textarabic{ـنـ}      & \textarabic{نـ}      \\
haaʾ         & h    & \textarabic{ه}      & \textarabic{ـه}       & \br{\textarabic{ـهـ}}   & \textarabic{هـ}      \\
\marginnote{The letters \textarabic{و} and \textarabic{ي} represent either a consonant or a long vowel and are transcribed accordingly.}%
waaw         & w/uu & \mc\br{\textarabic{و}}& \mc\br{\textarabic{ـو}} & \mc\br{\textarabic{ـو}} & \mc\br{\textarabic{و}} \\
yaaʾ         & y/ii & \br{\textarabic{ي}}   & \br{\textarabic{ـي}}    & \textarabic{ـيـ}      & \textarabic{يـ}      \\

   % \bottomrule
\end{tabular}



\section{Other Letters}

\bigskip

   \begin{tabular}{>{\itshape}l>{\itshape}cccccc} 
   % \toprule
   \multicolumn{1}{l}{\makebox[\alphcolw][l]{Name}} & \multicolumn{1}{l}{\makebox[\alphcolw][c]{Trans.}}   & \makebox[\alphcolw][c]{Isolated} & \makebox[\alphcolw][c]{Final} & \makebox[\alphcolw][c]{Medial} & \makebox[\alphcolw][c]{Initial}  \\
\midrule

\marginnote{Hamza is written with different "chairs" depending on surrounding vowels and its position in the word.}%
\multirow{5}*{hamza}                                                                                                                  & \multirow{5}*{ʾ} \rlap{\quad\rdelim\{{5}{3mm}} & \mc\textarabic{أ}                          & \mc\textarabic{ـأ}                                                              & \mc\textarabic{ـأ }     & \mc\textarabic{أ} \\
                                                                                                                                      &                                                & \mc\br{\textarabic{إ}}                          & \mc                                                                             & \mc                     & \mc\br{\textarabic{إ}} \\
                                                                                                                                      &                                                & \mc\br{\textarabic{ؤ}}                     & \mc\br{\textarabic{ـؤ}}                                                         & \mc\br{\textarabic{ـؤ}} & \mc\br{\textarabic{ؤ}} \\
                                                                                                                                      &                                                & \br{\textarabic{ئ}}                        & \br{\textarabic{ـئ}}                                                            & \textarabic{ـئـ}        & \textarabic{ئـ}\\
\marginnote{Hamza in the form of \textarabic{ء} does not connect to surrounding letters.}%
                                                                                                                                      &                                                & \mc\textarabic{ء}\vphantom{\textarabic{و}} & \mc\textarabic{ء}                                                               & \mc\textarabic{ء}       & \mc\textarabic{ء}\\
madda                                                                                                                                 & ʾaa                                            & \mc\textarabic{آ}                          & \mc\textarabic{ـآ}                                                              & \mc\textarabic{ـآ}      & \mc\textarabic{آ} \\
taaʾ marbuuTa                                                                                                                         & a/at                                           & \textarabic{ة}                             & \textarabic{ـة}                                                                 &                         & \\
\marginnote{The letters \textarabic{ة} and \textarabic{ى} only occur in word-final position. In words with possessive pronouns they transform to \textarabic{ـتـ} and~\textarabic{ـا} respectively.}%
alif maqSuura                                                                                                                         & aa                                             & \raiseyaa{\br{\textarabic{ى}}}             & \raiseyaa{\br{\textarabic{ـى}}}                                              \\
% \marginnote{\textarabic{لا} is the only compulsory ligature.}%
% \textarabic{[ل+ا]}                                                                                                                  & laa                                            & \mc\textarabic{لا}                          & \mc\textarabic{ـلا}                                                              & \mc\textarabic{ـلا}      & \mc\textarabic{لا}   \\
% \bottomrule
\end{tabular}

\vfill

\section{Vowels}

\bigskip

\begin{tabular}{
  >{\strut\itshape}l
  >{\hspace{2mm}\itshape}c@{~~}c@{\hspace{2em}}
  >{\hspace{2mm}\itshape}c@{~}c@{\hspace{2em}}
  >{\hspace{2mm}\itshape}cl
} 

  \multicolumn{1}{l}{Name} & 
  \multicolumn{2}{l}{Short}&
  \multicolumn{2}{l}{Long} &
  \multicolumn{2}{l}{Nunation}\\

  \midrule

  \marginnote{The \textarabic{ـا} in~\textarabic{ـًا} is silent.}%

  fatHa & a & \textarabic{ــَ} & aa & \textarabic{ـَا} & an & \textarabic{ــً}\rlap{\,/\,\textarabic{ـًا}}\\
  Damma & u & \textarabic{ــُ} & uu & \textarabic{ـُو} & un & \textarabic{ــٌ}                           \\
  kasra & i & \textarabic{ــِ} & ii & \textarabic{ـِي} & in & \textarabic{ــٍ}                           \\

  \addlinespace[1ex]

  
  \multicolumn{3}{l}{\itshape alif khanjariyya\marginnote{Alif khanjariyya is only used in the words {\newfontfamily\arabicfont[Script=Arabic]{Amiri}\upshape\textarabic{اللّٰه}}, \textarabic{لٰكِن}, \textarabic{هٰذا}, \textarabic{هٰذِهِ} and \textarabic{ذٰلِك}. Usually not printed even in voweled text.}}%
    & aa & \textarabic{ـٰ}\\

\end{tabular}

\vfill

\section{Miscellaneous}

\bigskip

\begin{tabular}{>{\strut\itshape}l>{\itshape}c>{\raggedright\normalsize\arraybackslash}p{10cm}} 
sukuun &    \textarabic{ــْ}& \begin{minipage}{\linewidth}\marginnote{Sukuun and shadda are only printed in voweled text.}%
Marks the absence of a vowel after a consonant. Not transcribed (\textarabic{دَرْس}~\textit{dars}).\end{minipage}\\
\addlinespace
\addlinespace
shadda &    \textarabic{ــّ}& \begin{minipage}{\linewidth}Marks doubling of a consonant. Vowel markers are written above or below \textit{shadda} rather than above or below the letter (\,\,\textarabic{ــَّ ــُّ ــِّ}\,). Transcribed as double consonant (\textarabic{مُدَرِّس}~\textit{mudarris}).\end{minipage} \\
% \bottomrule
\end{tabular}

\vfill\null

\end{document}
