% Search for SET to find things that need to be changed to to make another timeline out of this. 

\documentclass[
	a4paper % SET
	,landscape]{article}


\usepackage[dvipsnames]{xcolor}
\usepackage{/Users/xhalaa/dotfiles/mylatexstuff/ccbyandreas} \renewcommand{\ccyear}{2016, 2017, 2020} % SET. This calls some local macros on my computer to print the rights bit at the bottom. Remove this line unless you are on my computer.
\usepackage[top=2cm, bottom=2cm, left=2cm, right=2cm]{geometry}
\usepackage{etoolbox}
\usepackage{graphicx}
\usepackage{footnote} % to save and spew notes
\usepackage{ccicons}
\usepackage{calc}
\def\UrlFont{\rmfamily\itshape} % roman font in urls
\usepackage{ragged2e}
\usepackage{tikz}
\usepackage{varwidth} % nodes size with adjusted width
\usetikzlibrary{calc, backgrounds}
  \pgfdeclarelayer{background}
  \pgfdeclarelayer{foreground}
  \pgfsetlayers{background,main,foreground}
\usepackage{polyglossia}
\setmainlanguage{english}
\setotherlanguage{arabic}

% Fonts must be installed on your system. Called by name.
\usepackage{fontspec}
\setmainfont[Numbers=OldStyle]{Linux Libertine} % https://fontlibrary.org/en/font/linux-libertine
\setsansfont[Numbers=OldStyle]{Source Sans Pro ExtraLight} % https://github.com/adobe-fonts/source-sans

% Constants
\def\startce{600} % SET: start year
\def\fince{1550} % SET: en year

% Color to mix in with dynasty. 70%
\def\mixcolor{gray!60!white}

\frenchspacing
	
% Vertical position of beginning (leftmost edge) of name
\newlength\nameh
\setlength\nameh{6.1cm}

\newlength\dynastyheight
\setlength\dynastyheight{.4cm} % Thickness of dynasty bar
\newlength\dynastyvshift
\setlength\dynastyvshift{.55cm} % vertical shift of dynasty bar. How far the dynasty bar is shifted up from the main timeline

\newlength\dynwidth

% \grammarian[horizontal offset][east]{<name>}{<yod c.e.>}{<opus magnus>}
\newcommand\person[4][0]{
    % Name and yod on foreground
  \begin{pgfonlayer}{foreground}
    \path [draw] (#3,\nameh) ++ (#1,0) node [rotate=-90, anchor=west] (name) [fill=white, inner sep=0pt] {\makebox[2.5em][l]{#3}#2~\strut};
  \end{pgfonlayer}
    \draw [thin] (name.east) |- (#3,2) -- (#3,0) ;
    % Work below
    \node [font=\itshape, rotate=-90, anchor=west, yshift=#1] (work) at (#3,-2\baselineskip) {#4};
}

\newcommand\infobox[1]{%
  \draw[gray] let \p1 = (work) in node (infobox)
  [color=gray, anchor=north,draw=black, align=left, font=\scriptsize\sffamily]
   at (\x1,-5) {\begin{varwidth}{2.5cm}\RaggedRight#1\end{varwidth}};
  \draw (infobox) -- (work.east);
}


% \dynasty{<color>}{<name>}{<startyear c.e.>}{<endyear c.e.>}{<graphical level (>0)>}
\newcommand\dynasty[5]{%
\begin{pgfonlayer}{background}
    % bar
    \draw [yshift=\dynastyvshift+#5\dynastyheight, line width=\dynastyheight, color=#1!30!\mixcolor]
    (#3,0) -- (#4,0);
    % label
    \node [white, yshift=\dynastyvshift+#5\dynastyheight, font=\sffamily] 
    at ({#3+(#4-#3)/2},0) {#2};
\end{pgfonlayer}
}


% Command to mark areas on top of graph. The lines are slightly shortened (by the equivalent of y years) to leave a slight gap bet
% \phase{<name>}{<begin year>}{<end year>}
\newcommand{\phase}[3]{
    \draw [thin] (#2+2, \nameh+2mm) -- (#2+2, \nameh+4mm) -- (#3-2, \nameh+4mm) node [midway, above, align=center, font=\sffamily] {#1} -- (#3-2, \nameh+2mm);
}


\setlength\parindent{0pt}

\begin{document}
\thispagestyle{empty}

\begin{minipage}[t]{.4\textwidth} % SET: Width of textbox for title.
{\LARGE Timeline of Arab grammarians\\and their major works}\\ % SET: title. Add `\\` for manual line breaks.
	[\medskipamount] 
\textit{%
	v.1.6.1\\ % SET: version number. Can be removed.
	\today % SET: prints date of compilation. Type out a date manually if you want a fixed date.
	}
\end{minipage}
\hfill
\begin{minipage}[t]{7cm} % SET: Width and text box in upper right
  \setlength{\parfillskip}{0pt}
This timeline is based primarily on \textit{The Arabic Linguistic Tradition} by G. Bohas, \mbox{J.-P.} Guillaume, and D. Kouloughli (Georgetown University Press, 2006) and lists grammarians mentioned therein. % SET: Content of infobox
\end{minipage}


\vfill

\begin{tikzpicture}[x=\textwidth/(\fince-\startce)] % 1 step on x-axis defined as the text width divided total number of years from star year to en year as set above foe `\startce` (common era) to `\fince`.

\node at (650,\nameh-1ex) [anchor=north] {\textsc{yod}}; % SET: make year of death label. Can be removed and `yod` label changed

% Sort the list below with `sort n /{/`

		% SET. CANNOT OVERLAP!
		% Phases
		% Data collection, formation, theroeticlaelaboration, consolidiation, degeneration
		% \phase{<name>}{<begin year>}{<end year>}
    \phase{Precursors}{660}{770} 
    \phase{Early grammar}{770}{920} % p.8
    \phase{Codification\\and systematization}{920}{1100}
    \phase{Elaboration of forms\\of presentation}{1100}{1520}


    % SET
		% Dynasties
		% \dynasty{<color>}{<name>}{<startyear c.e.>}{<endyear c.e.>}{<graphical level}
		% Name color from the xcolor package.  Will be mixed with white and gray
		% Last argument is father 1 or 2. Set to choose height level to prevent overlap.
    \dynasty    {OliveGreen}   {Ummayads}      {661}    {750}      {1}
    %\dynasty   {blue}    {ar-Rāšhidun}   {632}    {661}      {1}
    \dynasty    {red}     {\hspace{4em}Abbasids}      {750}    {1258}     {1}
    \dynasty    {black}   {Fatimids}      {909}    {1171}     {2}
    \dynasty    {blue}     {Ottomans}      {1299}   {\fince}   {1}
    \dynasty    {brown}    {\hspace{2em}Mamluks}       {1250}   {1517}     {2}

		% SET
		% \person[horizontal offset][east]{<name>}{<yod c.e.>}{<opus magnus>}
    % Name and year and major work.
		% Optional argue met is a horizontal shift (negative or positive) manually inserted to prevent overlap. This will great a horizons line leading down to the main timeline to lead the eye. Takes a fair bit of manual tweaking to get nice results if you have many names close together.
    \person{Abū l-Aswad ad-Duʾalī}{688}{}
    \person{\llap{ʿ}Abd Allāh ibn Abī Isḥāq}{734}{} % p.1
    % \person[-1.7ex]{al-Xalil}{786}{Kitāb al-ʿAyn}
    \person{Sībawayhi}{798}{al-Kitāb} % p. 1. In Carter (2004:15) 796
    \person[-.5ex]{al-Farrāʾ}{822}{Maʿānī al-Qurʾān} % p.5
    \person[.7ex]{al-Axfaš al-Awsaṭ}{830}{Maʿānī al-Qurʾān} % p.5 835. EI 830
    % \person{al-Aṣmaʿī}{828}{}
    \person[-1.5ex]{al-Mubarrad}{898}{Kitāb al-Muqtaḍab}
    \person{Ṯaʿlab}{904}{} % p.7
    \person{Ibn as-Sarrāj}{928}{Kitāb al-Uṣūl} 
    % \infobox{\textit{First expression of final canonical grammar.}}
    \person{az-Zajjājī}{951}{Kitāb al-Iḍāḥ} % p 11
    \person{as-Sirāfī}{979}{Šarḥ kitāb Sībawayhi} % p.14 in passing
    \person{ar-Rummānī}{994}{Šarḥ kitāb Sībawayhi} % p.14 in passing
    \person[1ex]{Ibn Jinnī}{1002}{al-Xaṣāʾiṣ}
    % \person[-2ex]{Ibn Bābašāḏ}{1077}{}
    \person{al-Jurjānī}{1078}{Dalāʾil al-iʿjāz}
    % \infobox{\textit{with Jurjânî (d.1078) Arabic theory had reached its most sophisticated level} \citep{owens_foundations_1988}}
    \person{Ibn Maḍāʾ}{1196}{ar-Radd ʿalā an-nuḥāt} % p.58 1208. Versteegh (2013:208) 1196
    \person{Abū l-Barakāt al-Anbārī}{1181}{Lumaʿ al-adilla} % p.17
    \person{az-Zamaxšarī}{1143}{al-Mufaṣṣal} % p. 118
    \person{Ibn Yaʿīsh}{1245}{Šarḥ al-Mufaṣṣal}
    % \person{Ibn al-Ḥājib}{1249}{al-Kāfiya}
    % \infobox{\textit{Apogee of pedagogical grammars} \citep{carter_grammatical_2006}}
    \person[-2ex]{Ibn ʿUṣfur}{1271}{al-Mumtiʿ  fī t-taṣrīf} % p.73
    \person{Ibn Mālik}{1274}{al-Alfiyya}
    \person{al-Astarābāḏī}{1287}{Šarḥ al-Kāfiya}
    \person{Ibn Ājurrūm}{1323}{al-Ājurrūmiyya} % IE
    \person{Ibn Hišām}{1359}{Muġnī l-labīb} % p.14
    \person[1ex]{Ibn ʿAqīl}{1367}{Šarḥ al-Alfiyya} % p.14
    \person[-.5ex]{al-Ašmūnī}{1494}{Šarḥ al-Alfiyya} % p.14
    \person{as-Suyūṭī}{1505}{al-Muzhir} % IE
    

	% SET
  % Special arrow for Ottomans
  % One off thingie for the arrow continuing to the right.
	% Remove this paragraph if you do not want this arrow.
  \begin{pgfonlayer}{background}
      % bar
      \draw [ultra thin,yshift=\dynastyvshift, 
          xshift=-.1pt, fill=blue!30!\mixcolor,color=blue!30!\mixcolor]
      (\fince,.5\dynastyheight) 
      -- ++ (.5\dynastyheight,.5\dynastyheight)
      -- ++ (-.5\dynastyheight,.5\dynastyheight) 
      -- cycle;
  \end{pgfonlayer}


		% SET: this paragraph adds calculates and adds Hija years to the timeline. Remove the whole thing if you don't like it.
		\begin{pgfonlayer}{background}
		% hijri ticks
			\foreach \x in {0,100,...,900} 
				{\draw [red!70!gray] (.97*\x+622, 0) -- ++ (0, +.2) node [font=\bfseries,inner sep=0pt,anchor=south, color=red!70!gray] {\x\strut} ;}
			% hijri ticks on 50s
			\foreach \x in {50,150,...,900} 
			{\draw [red!70!gray] (.97*\x+622, 0) -- ++ (0, +.1);}
		\end{pgfonlayer}

	% Axis 
	\draw [->, very thick] (\startce,0) -- (\fince,0);

	% Axis ticks 
	\foreach \x in {
		              700% SET: first year for which you want a numbered tick. Be careful here, there cannot be any spaces in the foreach-list
									,800% SET: second year for which you want a numbered tick.
									,...% 
									,\fince % SET: Last year for which you want numbered tick. Can be a number or, as here, `\fince`, the end of the timelein as set above. All ticks in between will be generated by the `\foreach`-command.
								} 
	    {\draw (\x, 0) -- (\x,-.2) node [anchor=north, font=\bfseries,inner sep=0pt] {\x\strut};}

			% Secondary axis tick on 50s. (unlabeled)
	\foreach \x in {650% SET: first year for which you want a secondary tick.
								,700% SET: second year for which you want a numbered tick.
								,...%
								,1525% SET: Last year for which you want a secondary tick.
								}
	    {\draw (\x, 0) -- (\x,-.1);}

  % SET
	% Label for star of timeline.
	% Delete this paragraph to remove start label.
	\draw [very thick] (622, 0) -- (622, -.5) % 622 here (in two places) is to specify the position of the star label.
	  node [rotate=-90, anchor=west] 
		{\textbf{\textit{al-Hijra}}~(622)}; % Text label for start

\end{tikzpicture}

\vfill\null
\end{document}
