\documentclass{article}
\usepackage{tikz}
\usetikzlibrary{backgrounds}
\usepackage[a4paper,landscape, hmargin=5.5cm, vmargin=3cm ]{geometry}
\usepackage{xcolor}
\usepackage{hhline}
\usepackage{letltxmacro} % Patch commands with let
\usepackage{etoolbox}
\usepackage{newunicodechar}
\usepackage{colortbl}
\usepackage{microtype}
\usepackage{calc}
\usepackage{booktabs}
\usepackage{multicol}
\usepackage{tabularx}
\usepackage{threeparttable}
\usepackage{array}
\usepackage{\string ~/mylatexstuff/ccbyandreas}
\usepackage{polyglossia}
\setmainlanguage{english}
\setotherlanguage{arabic}
\newfontfamily\arabicfont[Script=Arabic, Scale=1.4]
% {Scheherazade}
{Lateef}
% {DejaVu Sans Mono}
% {Amiri}
\setmainfont[Numbers=OldStyle]{Linux Libertine O}
% \setmainfont[Numbers=OldStyle]{Source Sans Pro}
\frenchspacing
\setlength\tabcolsep{5pt}

% Make Arabic text take no vertical space to remove extra line height
\LetLtxMacro{\oldtextarabic}{\textarabic}
\renewcommand{\textarabic}[1]{\smash{\oldtextarabic{#1}}}

\setlength\parindent{0pt}


\pagestyle{empty}

\catcode`\_=\active

\let\root\undefined
\def_{\root}

% \newcommand\root{\raisebox{-.2ex}{\tikz \draw [color=white, fill=black!15](0,-.2ex) circle (.7ex);}}
\newcommand\root{\raisebox{-.2ex}{%
  \tikz\draw [color=gray, fill=black!15](0,-.2ex) rectangle (1.2ex,1.4ex);%
}}

\newcommand{\explaingrayrow}{%
  The gray row shows the shortened form of 1s past tense applicable also to 2ms, 2fs, 1pl, 2mpl, and 3fpl (i.e with the ending \textarabic{ـتَ}\,, \textarabic{ـتِ}\,, \textarabic{ـنا}\,, \textarabic{ـتم}\,, or~\textarabic{ـن}\,). Example verbs are taken from {\upshape A Frequency Dictionary of Arabic} by Tim Buckwalter and Dilworth Parkinson (Routledge 2011).%
}

\let\d\undefined
\let\k\undefined
\let\s\undefined
\let\sf\undefined
\let\do\undefined

\renewcommand\arraystretch{2}


\newcommand{\n}{_}% none
\newcommand{\f}{_َ}% fatha
\newcommand{\d}{_ُ}% damma
\newcommand{\k}{_ِ}% kasra
\newcommand{\kn}{_ٍ}% kasra nunation
\newcommand{\skn}{_ٍّ}% shadda kasra nunation
\newcommand{\sf}{_َّ}% shadda fatha
\newcommand{\sd}{_ُّ}% shadda damma
\newcommand{\sk}{_ِّ}% shadda kasra
\newcommand{\s}{_ْ}% sukuun
\newcommand{\sh}{_ّ}% sshadda

\newcommand{\dou}{\underline{\root\root}}

\newcolumntype{Y}{>{\centering\arraybackslash}X}
\newcommand{\rc}{\rowcolor{black!10}}

\begin{document}

\center

% title
\centerline{\huge \scshape Hollow Verbs Form I--X}

\bigskip
\centerline{\footnotesize\itshape \today}

\vfill

 \begin{minipage}{\textwidth}
  \renewcommand\baselinestretch{1.2}
  \setlength\parfillskip{0pt}
  \itshape
  \begin{multicols}{3}
    Hollow verbs are those with \textarabic{و} or~\textarabic{ي} as their middle root consonant. In form~\textsc{i} there are two different inflectional classes for verbs with the middle root~\textarabic{و} and one for verbs with the middle root~\textarabic{ي}. In forms~\textsc{ii--x} all verbs with hollow roots are inflected the same way regardless of middle root. Note that hollow verbs in forms \textsc{ii--vi} and~\textsc{ix} are inflected exactly like strong verbs.
    \explaingrayrow 
  \end{multicols}
 \end{minipage}

 \vfill
 \vfill

\newcolumntype{A}{>{\begin{}r<{\end{Arabic}}}Arabic}



\begin{Arabic}
  \begin{tabularx}{\linewidth}{lYYYYYYYYYYYc}
    \toprule
    &\large \textenglish{I \itshape w-a}    &\large \textenglish{I \itshape w-b}&\large\textenglish{I \itshape y}&\large \textenglish{II} &\large \textenglish{III}               &\large \textenglish{IV} &\large \textenglish{V} &\large \textenglish{VI} &\large \textenglish{VII} &\large \textenglish{VIII} &\large \textenglish{IX} &\large \textenglish{X} \\

\midrule
\hhline{~---}
\textenglish{Past}        & \multicolumn{3}{|c|}{\n ـا\f}                       & \f \sf \f     & \n ـا\f\f       & أَ\n ـا\f    & تَـ\f\sf\f    & تَـ\n ـا\f\f    & اِنْـ\f ـا\f  & اِ\s ـتَـ\n ـا\f  & اِ\s\f\sf       & اِسْتَـ\n ـا\f \\   
\hhline{~---}                                                                                                                                                                                                                    
\rc\textenglish{Past 1s}    & \d\s ـتُ      & \k\s ـتُ    & \k\s ـتُ               & \f \sf \s ـتُ  & \n ـا\f\s ـتُ    & أَ\f \s ـتُ   & تَـ\f\sf\s ـتُ & تَـ\n ـا\f\s ـتُ & اِنْـ\f\s ـتُ  & اِ\s ـتَـ\s ـتُ    & اِ\s\f\f\s ـتُ   & اِسْتَـ\f \s ـتُ \\  
\textenglish{Present}       & يَـ\d ـو\n   & يَـ\n ـا\n  & يَـ\k ـيـ\n             & يُـ\f \sk \n   & يُـ\n ـا\k \n    & يُـ\k ـيـ\n  & يَتَـ\f\sf\n   & يَتَـ\n ـا\f\n   & يَنْـ\n ـا\n  & يَـ\s ـتا\n      & يَـ\s\f\sh      & يَسْتَـ\k ـيـ\n \\  
\hhline{~---}                                                                                                                                                                                                                    
\textenglish{\textit{maṣdar}} & \multicolumn{3}{|c|}{\textenglish{Irregular}}   & تَـ\s \k ـيـ\n & مُـ\n ـا\f \n ـة & إِ\n ـا\n ـة & تَـ\f\sd\n    & تَـ\n ـا\d\n    & اِنْـ\k ـيا\n & اِ\s ـتِـ\k ـيا\n & اِ\s\k\n ـا\n   & اِسْتِـ\n ـا\f ـة \\
\hhline{~---}                                                                                                                                                                                                                    
\textenglish{Act. part.}    & \multicolumn{3}{|c|}{\n ـائِـ\n}                   & مُـ\f \sk \n   & مُـ\n ـا\k \n    & مُـ\n ـِيـ\n  & مُتَـ\f\sk\n   & مُتَـ\n ـا\k\n   & مُنْـ\f ـا\n  & مُـ\s ـتا\n      & مُـ\s\f\sh      & مُسْتَـ\k ـيـ\n \\  
\hhline{~---}                                                                                                                                                                                                                    
\textenglish{Pass. part.}   & مَـ\d ـو\n & مَـ\d ـو\n  & مَـ\k ـيـ\n               & مُـ\f\sf \n    & مُـ\n ـا\f\n     & مُـ\n ـا\n   & مُتَـ\f\sf\n   & مُتَـ\n ـا\f\n   & مُنْـ\f ـا\n  & مُـ\s ـتا\n      &                & مُسْتَـ\n ـا\n \\   
\midrule                                                                                                                                                                                                                         
\textenglish{Example} & قال & نام &عاش                                          & حول           & حاول            & أراد        & تجوز         & تناول          & انهار       & احتاج           & اسود           & استطاع\\         
\addlinespace
\bottomrule

        \end{tabularx}     
 \end{Arabic}              

 \vfill
 \null

\newpage

\centerline{\huge \scshape Defective Verbs Form I--X}

\bigskip
\centerline{\footnotesize\itshape \today}

\vfill

\begin{minipage}{\textwidth}
  \renewcommand\baselinestretch{1.2}
  \setlength\parfillskip{0pt}
  \itshape
  \begin{multicols}{3}
    Defective verbs are those that have \textarabic{و} or~\textarabic{ي} as their third root consonant. In form~\textsc{i} there are two different inflectional classes for verbs with the final root~\textarabic{ي}, and one for verbs with the final root~\textarabic{و}. In forms~\mbox{\textsc{ii--x}} all defective verbs are inflected the same way regardless of whether the final root is \textarabic{و} or~\textarabic{ي}. The nunation~\textarabic{ـٍ} in active participles and \textit{maṣdar}s is replaced with~\textarabic{ـِي} when the word is definite and in construct state (e.g~\textarabic{قاضٍ، القاضي، قاضي بغداد}). Passive participles with~\textarabic{ى} are traditionally described as having an ending~\textarabic{ـً}\,. This is however rarely enunciated or produced in print. Defective verbs in form~\textsc{ix} are unattested in modern Arabic and very rare in Classical Arabic (see Wright's \textup{Grammar}, vol.i, ¶59).
    \explaingrayrow
  \end{multicols}
\end{minipage}

\vfill

\begin{Arabic}
  \begin{tabularx}{\textwidth}{lYYYYYYYYYYYc}

    \toprule
\Large
    &\large \textenglish{I \itshape y-a}    &\large \textenglish{I \itshape y-b}&\large\textenglish{I \itshape w}&\large \textenglish{II} &\large \textenglish{III}               &\large \textenglish{IV} &\large \textenglish{V} &\large \textenglish{VI} &\large \textenglish{VII} &\large \textenglish{VIII}  &\large \textenglish{X} \\

\midrule
\textenglish{Past}          & \f\n ـى        &\f\k ـيَ      & \f\n ـا            & \f \sf ـى     & \n ـا\f ـى      & أَ\s\n ـى     & تَـ\f\sh ـى      & تَـ\n ـا\n ـى       & اِنْـ\f\n ـى      & اِ\s ـتَـ\n ـى    & اِسْتَـ\s \n ـى \\   
\rc\textenglish{Past 1s}    & \f\f\s ـتُ      & \f\k ـيتُ    & \f\f\s ـتُ          & \f \sf ـيْتُ    & \n ـا\f ـيْتُ     & أَ\s\f ـيْتُ    & تَـ\f\sf ـيْتُ     & تَـ\n ـا\f ـيْتُ      & اِنْـ\f\f ـيْتُ     & اِ\s ـتَـ\f ـيْتُ   & اِسْتَـ\n\f ـيْتُ \\  
\textenglish{Present}       & يَـ\s\k ـي      & يَـ\s\n ـى   & يَـ\s\d ـو          & يُـ\f \sk \n   & يُـ\n ـا\k ـي    & يُـ\k ـيـ\n   & يَتَـ\f\sh ـى     & يَتَـ\n ـا\n ـى      & يَنْـ\f\k ـي      & يَـ\s ـتَـ\k ـي   & يَسْتَـ\s\k ـي\\  
\hhline{~---}
\textenglish{\textit{maṣdar}} & \multicolumn{3}{|c|}{\textenglish{Irregular}}   & تَـ\s \k ـية   & مُـ\n ـا\n ـاة   & إِ\s\n ـاء    & تَـ\f\skn        & تَـ\n ـا\kn         & اِنْـ\k\n ـاء     & اِ\s ـتِـ\n ـاء   & اِسْتِـ\s\n ـاء \\
\hhline{~---}
\textenglish{Act. part.}    & \multicolumn{3}{|c|}{\n ـا\kn}                    & مُـ\f \skn     & مُـ\n ـا\kn      & مُـ\s\kn      & مُتَـ\f\skn       & مُتَـ\n ـا\kn        & مُنْـ\f\kn        & مُـ\s ـتَـ\kn     & مُسْتَـ\s\kn \\  
\hhline{~---}
\textenglish{Pass. part.}   & مَـ\s\k ـيّ & مَـ\s\k ـيّ &مَـ\s\d ـوّ                  & مُـ\f\sh ـى    & مُـ\n ـا\n ـى    & مُـ\s\n ـى    & مُتَـ\f\sh ـى      & مُتَـ\n ـا\n ـى     & مُنْـ\f\n ـى      & مُـ\s ـتَـ\n ـى   & مُسْتَـ\s\n ـى\\   
\midrule                                                                                                                                                                                                          
\textenglish{Example} & مشى & بقي &دعا                                          & صلى           & نادى            & أعطى         & تمنى            & توالى              & انبغى           & اشترى           & استدعى\\         
\addlinespace
\bottomrule

        \end{tabularx}
 \end{Arabic}              

  % \begin{tablenotes}
  %   \footnotesize
  % \item[*] Common forms include~~\textarabic{\k\n ـا\n ـة}~,~~\textarabic{\f\s\n}~~and~~\textarabic{\d\d ـو\n}~. 
  %   \item[**] For verbs denoting conflict alternatively \textarabic{\k\n ـا\n} .
  % \end{tablenotes}



\vfill
\null

\end{document}
