\documentclass[oneside]{article}
% \usepackage[all, placement=bottom, scale=1, color=black, opacity=1, vshift=1cm]{background}
\usepackage{/Users/xhalaa/dotfiles/mylatexstuff/ccbyandreas}
  % Get this file https://github.com/andreasmhallberg/dotfiles/blob/master/mylatexstuff/ccbyandreas.sty. Place in the same directory as this file and change the command above to `\usepackage{ccbyandreas}`.
\renewcommand{\ccyear}{2017,2023}
\PassOptionsToPackage{hidelinks}{hyperref}
\renewcommand\UrlFont\itshape
\usepackage{array}
\usepackage[
  left=2.3cm
  , right=6cm
  , marginparwidth=4cm
  , vmargin={3cm}
% , showframe
]{geometry}
\usepackage[compact, sc]{titlesec}
\titlelabel{\thetitle.\ }
\usepackage{booktabs}
\usepackage{microtype}
\usepackage{bigdelim}
\usepackage{tabularx}
\usepackage{calc}
\usepackage{xcolor}
\usepackage{marginnote}
\usepackage{colortbl} % To highlight cells in tables. See \mc command.
% Mark single cells
\newcommand\mc{\cellcolor{black!10}}
\usepackage{polyglossia}
\setmainfont[Numbers=OldStyle]{Linux Libertine}
\setmainlanguage{english}
\setotherlanguage{arabic}
\usepackage{multirow}

\renewcommand*{\marginfont}{\normalsize\itshape\sloppy}
\renewcommand*{\raggedleftmarginnote}{}
\renewcommand*{\raggedrightmarginnote}{}

\setlength\parindent{0pt}

% command to raise letters yā in table for better viual palance and to avoid clashing
\newcommand{\raiseyā}[1]{\raisebox{2pt}{#1}}

\newfontfamily\arabicfont[Script=Arabic,Scale=1.5]{Scheherazade}
\newfontfamily\arabicfontsf[Script=Arabic]{Amiri} % use sf for ligautres

\newcommand\ligaturetable{%
\renewcommand\arraystretch{1.2}%
    \upshape
    \framebox{%
\begin{tabularx}{\linewidth}{
    >{\sf}r@{~\color{gray}$\rightarrow$~}
    >{\sf}r
    >{\sf}r@
    {~~}>{\itshape}X@{}}
    \multicolumn{4}{l}{\textbf{Common ligatures}}\\
    \addlinespace
    \textarabic{لـ\char"200D ا}     & \textarabic{لا}\rlap{*\ }    & \textarabic{سلام}     & salām   \\  
    \textarabic{لـمـ}                  & \textarabic{لمـ}  & \textarabic{المسلم} & \mbox{al-muslim} \\ 
  \textarabic{فـي}                 & \textarabic{في}   &                     & fī \\           
\textarabic{يـحـ}                  & \textarabic{يحـ}  & \textarabic{يحيا}   & yaḥyā  \\   
\textarabic{تـجـ}                  & \textarabic{تجـ}  & \textarabic{تجارة}  & tijāra \\   
\textarabic{مـحـ}                  & \textarabic{محـ}  & \textarabic{محل}    & maḥall\\     
\textarabic{مـمـ}                  & \textarabic{ممـ}  & \textarabic{ممل}    & mumill\\     
\textarabic{الـلـه}                & \textarabic{الله} &                     & allāh \\           
\addlinespace
\multicolumn{4}{l}{* Non-optional}\\



\end{tabularx}%
}}

\pagestyle{empty}

\frenchspacing

\begin{document}
\Large

% columnwidth
    \newlength\alphcolw
    \setlength\alphcolw{3em}

% baselinerule
    \newcommand{\br}[1]{%
    % \rlap{\hspace{-.3em}\color{red}\rule{.5\alphcolw}{.1pt}}%
        \rlap{\color{red}\hspace{-3pt}\rule{\widthof{#1}+8pt}{.2pt}}#1%
}


{\fontspec{Linux Libertine}\huge\itshape The Arabic writing system}%
\marginnote{This chart uses the transcription system of \href{https://www.loc.gov/catdir/cpso/romanization/arabic.pdf}{Library of Congress}.
For version with other transcription systems, see \url{http://andreasmhallberg.github.io}.}[-1cm]
\\{\small\itshape\today}

\vfill

\section{The Alphabet}

\bigskip

\begin{tabular}{>{\strut\itshape}l>{\itshape}cccccc} 
   % \toprule
   \multicolumn{1}{l}{%
   \makebox[\alphcolw][l]{Name}} & \multicolumn{1}{l}{\makebox[\alphcolw][c]{Trans.}}  & \makebox[\alphcolw][c]{Isolated}& \makebox[\alphcolw][c]{Final} & \makebox[\alphcolw][c]{Medial} & \makebox[\alphcolw][c]{Initial}   \\
\midrule

       \marginnote{Gray \colorbox{black!10}{background} marks letters that do not connect forward. A following letter takes the initial or isolated form.}%
alif                                                                                                                                                              & ā   & \mc\textarabic{ا}      & \mc\textarabic{ـا}      & \mc\textarabic{ـا }     & \mc\textarabic{ا}    \\
bāʾ                                                                                                                                                              & b    & \textarabic{ب}         & \textarabic{ـب}         & \textarabic{ـبـ}        & \textarabic{بـ}      \\
tāʾ                                                                                                                                                              & t    & \textarabic{ت}         & \textarabic{ـت}         & \textarabic{ـتـ}        & \textarabic{تـ}      \\
thāʾ                                                                                                                                                             & th   & \textarabic{ث}         & \textarabic{ـث}         & \textarabic{ـثـ}        & \textarabic{ثـ}      \\
\marginnote{The baseline is marked with a \br{red line} on descending letter forms.}%
jīm                                                                                                                                                              & j    & \br{\textarabic{ج}}    & \br{\textarabic{ـج}}    & \textarabic{ـجـ}        & \textarabic{جـ}      \\
ḥāʾ                                                                                                                                                              & ḥ    & \br{\textarabic{ح}}    & \br{\textarabic{ـح}}    & \textarabic{ـحـ}        & \textarabic{حـ}      \\
khāʾ                                                                                                                                                             & kh   & \br{\textarabic{خ}}    & \br{\textarabic{ـخ}}    & \textarabic{ـخـ}        & \textarabic{خـ}      \\
dāl                                                                                                                                                              & d    & \mc\textarabic{د}      & \mc\textarabic{ـد}      & \mc\textarabic{ـد}      & \mc\textarabic{د}    \\
dhāl                                                                                                                                                             & dh   & \mc\textarabic{ذ}      & \mc\textarabic{ـذ}      & \mc\textarabic{ـذ}      & \mc\textarabic{ذ}    \\
rāʾ                                                                                                                                                              & r    & \mc\br{\textarabic{ر}} & \mc\br{\textarabic{ـر}} & \mc\br{\textarabic{ـر}} & \mc\br{\textarabic{ر}} \\
zāy                                                                                                                                                              & z    & \mc\br{\textarabic{ز}} & \mc\br{\textarabic{ـز}} & \mc\br{\textarabic{ـز}} & \mc\br{\textarabic{ز}} \\
sīn                                                                                                                                                              & s    & \br{\textarabic{س}}    & \br{\textarabic{ـس}}    & \textarabic{ـسـ}        & \textarabic{سـ}      \\
shīn                                                                                                                                                             & sh   & \br{\textarabic{ش}}    & \br{\textarabic{ـش}}    & \textarabic{ـشـ}        & \textarabic{شـ}      \\
\marginnote{\textarabic{ص}, \textarabic{ض}, \textarabic{ط} and \textarabic{ظ} are the so called emphatic letters and affect the quality of nearby vowels.}[-1ex]%
ṣād                                                                                                                                                              & ṣ    & \br{\textarabic{ص}}    & \br{\textarabic{ـص}}    & \textarabic{ـصـ}        & \textarabic{صـ} \rlap{\color{gray}\hspace{1em}\ldelim]{4}{3mm}}     \\
ḍād                                                                                                                                                              & ḍ    & \br{\textarabic{ض}}    & \br{\textarabic{ـض}}    & \textarabic{ـضـ}        & \textarabic{ضـ}      \\
ṭāʾ                                                                                                                                                              & ṭ    & \textarabic{ط}         & \textarabic{ـط}         & \textarabic{ـطـ}        & \textarabic{طـ}      \\
\marginnote{\small\ligaturetable}%
 ẓāʾ                                                                                                                                                            & ẓ   & \textarabic{ظ}         & \textarabic{ـظ}         & \textarabic{ـظـ}        & \textarabic{ظـ}      \\
 \llap{ʿ}ayn                                                                                                                                                      & ʿ    & \br{\textarabic{ع}}    & \br{\textarabic{ـع}}    & \textarabic{ـعـ}        & \textarabic{عـ}      \\
ghayn                                                                                                                                                             & gh   & \br{\textarabic{غ}}    & \br{\textarabic{ـغ}}    & \textarabic{ـغـ}        & \textarabic{غـ}      \\
fāʾ                                                                                                                                                              & f    & \textarabic{ف}         & \textarabic{ـف}         & \textarabic{ـفـ}        & \textarabic{فـ}      \\
qāf                                                                                                                                                              & q    & \br{\textarabic{ق}}    & \br{\textarabic{ـق}}    & \textarabic{ـقـ}        & \textarabic{قـ}      \\
kāf                                                                                                                                                              & k    & \textarabic{ك}         & \textarabic{ـك}         & \textarabic{ـكـ}        & \textarabic{كـ}      \\
lām                                                                                                                                                              & l    & \br{\textarabic{ل}}    & \br{\textarabic{ـل}}    & \textarabic{ـلـ}        & \textarabic{لـ}      \\
mīm                                                                                                                                                              & m    & \br{\textarabic{م}}    & \br{\textarabic{ـم}}    & \textarabic{ـمـ}        & \textarabic{مـ}      \\
nūn                                                                                                                                                              & n    & \br{\textarabic{ن}}    & \br{\textarabic{ـن}}    & \textarabic{ـنـ}        & \textarabic{نـ}      \\
hāʾ                                                                                                                                                              & h    & \textarabic{ه}         & \textarabic{ـه}         & \br{\textarabic{ـهـ}}   & \textarabic{هـ}      \\
\marginnote{The letters \textarabic{و} and \textarabic{ي} represent either a consonant or a long vowel and are transcribed accordingly.}%
wāw                                                                                                                                                              & w/ū & \mc\br{\textarabic{و}} & \mc\br{\textarabic{ـو}} & \mc\br{\textarabic{ـو}} & \mc\br{\textarabic{و}} \\
yāʾ                                                                                                                                                              & y/ī & \br{\textarabic{ي}}    & \br{\textarabic{ـي}}    & \textarabic{ـيـ}        & \textarabic{يـ}      \\

   % \bottomrule
\end{tabular}



\section{Other Letters}

\bigskip

   \begin{tabular}{>{\itshape}l>{\itshape}cccccc} 
   % \toprule
   \multicolumn{1}{l}{\makebox[\alphcolw][l]{Name}} & \multicolumn{1}{l}{\makebox[\alphcolw][c]{Trans.}}   & \makebox[\alphcolw][c]{Isolated} & \makebox[\alphcolw][c]{Final} & \makebox[\alphcolw][c]{Medial} & \makebox[\alphcolw][c]{Initial}  \\
\midrule

\marginnote{Hamza is written with different "chairs" depending on surrounding vowels and its position in the word.}%
\multirow{5}*{hamza}                                                                                                                  & \multirow{5}*{ʾ} \rlap{\quad\rdelim\{{5}{3mm}} & \mc\textarabic{أ}                          & \mc\textarabic{ـأ}                                                              & \mc\textarabic{ـأ }     & \mc\textarabic{أ} \\
                                                                                                                                      &                                                & \mc\br{\textarabic{إ}}                          & \mc                                                                             & \mc                     & \mc\br{\textarabic{إ}} \\
                                                                                                                                      &                                                & \mc\br{\textarabic{ؤ}}                     & \mc\br{\textarabic{ـؤ}}                                                         & \mc\br{\textarabic{ـؤ}} & \mc\br{\textarabic{ؤ}} \\
                                                                                                                                      &                                                & \br{\textarabic{ئ}}                        & \br{\textarabic{ـئ}}                                                            & \textarabic{ـئـ}        & \textarabic{ئـ}\\
\marginnote{Hamza in the form of \textarabic{ء} does not connect to surrounding letters.}%
                                                                                                                                      &                                                & \mc\textarabic{ء}\vphantom{\textarabic{و}} & \mc\textarabic{ء}                                                               & \mc\textarabic{ء}       & \mc\textarabic{ء}\\
madda                                                                                                                                 & ʾā                                            & \mc\textarabic{آ}                          & \mc\textarabic{ـآ}                                                              & \mc\textarabic{ـآ}      & \mc\textarabic{آ} \\
tāʾ marbūṭa                                                                                                                         & a/at                                           & \textarabic{ة}                             & \textarabic{ـة}                                                                 &                         & \\
\marginnote{The letters \textarabic{ة} and \textarabic{ى} only occur in word-final position. In words with possessive pronouns they transform to \textarabic{ـتـ} and~\textarabic{ـا} respectively.}%
alif maqṣūra                                                                                                                         & ā                                             & \raiseyā{\br{\textarabic{ى}}}             & \raiseyā{\br{\textarabic{ـى}}}                                              \\
% \marginnote{\textarabic{لا} is the only compulsory ligature.}%
% \textarabic{[ل+ا]}                                                                                                                  & lā                                            & \mc\textarabic{لا}                          & \mc\textarabic{ـلا}                                                              & \mc\textarabic{ـلا}      & \mc\textarabic{لا}   \\
% \bottomrule
\end{tabular}

\vfill

\section{Vowels}

\bigskip

\begin{tabular}{
  >{\strut\itshape}l
  >{\hspace{2mm}\itshape}c@{~~}c@{\hspace{2em}}
  >{\hspace{2mm}\itshape}c@{~}c@{\hspace{2em}}
  >{\hspace{2mm}\itshape}cl
} 

  \multicolumn{1}{l}{Name} & 
  \multicolumn{2}{l}{Short}&
  \multicolumn{2}{l}{Long} &
  \multicolumn{2}{l}{Nunation}\\

  \midrule

  \marginnote{The \textarabic{ـا} in~\textarabic{ـًا} is silent.}%

  fatḥa & a & \textarabic{ــَ} & ā & \textarabic{ـَا} & an & \textarabic{ــً}\rlap{\,/\,\textarabic{ـًا}}\\
  ḍamma & u & \textarabic{ــُ} & ū & \textarabic{ـُو} & un & \textarabic{ــٌ}                           \\
  kasra & i & \textarabic{ــِ} & ī & \textarabic{ـِي} & in & \textarabic{ــٍ}                           \\

  \addlinespace[1ex]

  
  \multicolumn{3}{l}{\itshape alif khanjariyya\marginnote{Alif khanjariyya is only used in the words {\fontspec[Script=Arabic]{Amiri}\upshape\textarabic{اللّٰه}}, \textarabic{لٰكِن}, \textarabic{هٰذا}, \textarabic{هٰذِهِ} and \textarabic{ذٰلِك}. Usually not printed even in voweled text.}}%
    & ā & \textarabic{ـٰ}\\

\end{tabular}

\vfill

\section{Miscellaneous}

\bigskip

\begin{tabular}{>{\strut\itshape}l>{\itshape}c>{\raggedright\normalsize\arraybackslash}p{10cm}} 
sukūn &    \textarabic{ــْ}& \begin{minipage}{\linewidth}\marginnote{Sukūn and shadda are only printed in voweled text.}%
Marks the absence of a vowel after a consonant. Not transcribed (\textarabic{دَرْس}~\textit{dars}).\end{minipage}\\
\addlinespace
\addlinespace
shadda &    \textarabic{ــّ}& \begin{minipage}{\linewidth}Marks doubling of a consonant. Vowel markers are written above or below \textit{shadda} rather than above or below the letter (\,\,\textarabic{ــَّ ــُّ ــِّ}\,). Transcribed as double consonant (\textarabic{مُدَرِّس}~\textit{mudarris}).\end{minipage} \\
% \bottomrule
\end{tabular}

\vfill\null

\end{document}
