\documentclass[oneside,a4paper]{article}

\frenchspacing

\usepackage[
    % left=2.5cm
  % , right=1.5cm
  % , marginparsep=1cm
  % , marginparwidth=4cm, includemp
  % , bottom={4cm}
  % , showframe
]{geometry}

\usepackage{calc}

% \usepackage[explicit]{titlesec}
% \titleformat
% {\section}%command
% [block]%shape. block: paragraph without formatting.
% {\LARGE\bfseries}%format
% {\marginpar{\raggedleft\addfontfeatures{Scale=4}\upshape\huge%
%   \liningnums{\raisebox{-1ex}{\color{black!30}\thesection}}%
%   }%
% }%label
% {0pt}%sep between number and title
% {\reversemarginpar#1}%before
% [\normalmarginpar]%after

\usepackage{xcolor}

\usepackage{needspace}
% \usepackage{titleps}
% \newpagestyle{mystyle}{\sethead{}{\bfseries\sffamily\Large\color{black!20} DRAFT --- DO NOT DISTRIBUTE}{}\setfoot{}{}{\thepage}}
\usepackage[hyperfootnotes=false,hidelinks]{hyperref}

% TYPOGRAPHY
\usepackage{microtype}
\usepackage{booktabs}
\usepackage{tabularx}
\usepackage{multicol}

\usepackage{etoolbox}

\BeforeBeginEnvironment{multicols}{\savenotes\begin{minipage}{\textwidth}}
\AfterEndEnvironment{multicols}{\end{minipage}\spewnotes}

\usepackage{ragged2e}
\usepackage{array}
\usepackage{multirow}
\renewcommand{\multirowsetup}{\sloppy}
% \usepackage{/Users/xhalaa/dotfiles/mylatexstuff/ccbyandreas.sty} \renewcommand{\ccyear}{2020}

\newcolumntype{Y}{>{\RaggedRight\arraybackslash}X} 
\newcolumntype{Z}{>{\RaggedLeft \arraybackslash}X} 

\usepackage{footnote} % provides \savenotes and \spewnotes
% \makesavenoteenv{tabular}
\usepackage[bottom]{footmisc}

% \appto{\section}{\nopagebreak}
% \appto{\subsection}{\nopagebreak}

\renewcommand{\arraystretch}{1.1}

\preto{\subsection}{\goodbreak}

\usepackage{marginnote}
% Justify left and right margin notes
\renewcommand*{\raggedleftmarginnote}{\raggedleft}
\renewcommand*{\raggedrightmarginnote}{\raggedleft}
\renewcommand{\marginfont}{\upshape}

% \renewcommand{\footnote}[1]{%
%   \textsuperscript{\textup{\refstepcounter{footnote}}}%
%   \marginnote{\textup{\llap{\textsuperscript{\thefootnote}}}#1}%
%   }
% \preto{\footnote}{\marginnotemark}

\newcommand{\marginnotemark}{\textsuperscript{\thefootnote}}

% \preto{\section}{\bigskip\hrule}

\usepackage{enumitem}
\setlist{nosep}

\usepackage{xcolor}

\raggedbottom

% LANGUAGES
\usepackage{polyglossia}
\setmainfont[Numbers=OldStyle]{Linux Libertine O}
\setmainlanguage{english}
\setotherlanguage{arabic}
\newfontfamily\arabicfont[Script=Arabic,Scale=1.4]{Lateef}

\setlength{\parindent}{0pt}
\setlength{\parskip}{\medskipamount}

\newcommand{\trans}[1]{\textit{\hyphenpenalty=10000#1}}

% Images
% Metadata

\title{Syrian Arabic grammatical\\summary reference}
\author{Andreas Hallberg}


% leftward compensation for inital ayn and hamza in tabulars
\newcommand{\ayntab}{\null\hspace{-.2em}}

% pagerefs to cowell
\newcommand{\page}[1]{\marginnote{\normalsize{\upshape(#1)}}}

\begin{document}

{\Huge\itshape Syrian Arabic grammatical\\[\medskipamount]reference sheet}

\bigskip

% \marginnote{
% \tt\RaggedRight
% TODO:\\
% - uses of participles\\
% - defective verbs\\
% - double verbs\\
% }
Andreas Hallberg\\
\today

\bigskip


\hrule
\setlength{\columnsep}{1em}
\begin{multicols}{3}
  \setlength\parfillskip{0pt}
\itshape
This is a brief grammatical summary of the variety of Arabic spoken in the Damascus area in Syria. It is intended to be used for quick reference. For a more detailed grammatical description, see \textit{A~Reference Grammar of Syrian Arabic} (Cowell, Georgetown University Press, 2005), to which page references are provided in the margin. The transcription system is based on \textit{Alif Baa} (Brustad et~al., Georgetown University Press, 2010), extended with the vowels \trans{e/ee} and~\trans{o/oo}. The Arabic writing represent written Syrian Arabic vernacular, as used in text messaging, advertisement, etc. There is some variation in this form of writing, especially in the degree to which Standard Arabic orthography is retained.
\end{multicols}

\bigskip
\hrule

\setlength{\columnsep}{30pt}

\section{Phonemics}

\subsection{Consonants}


\begin{multicols}{2}

  \begin{tabularx}{\linewidth}{X>{\itshape}c>{\itshape}cX>{\itshape}c>{\itshape}cX}

&ʾ    & \textarabic{ا}&& D   & \textarabic{ض}   &    \\
&b    & \textarabic{ب}&& T    & \textarabic{ط}  &     \\
&t    & \textarabic{ت}&& (DH) & \textarabic{ظ}  &     \\
&(th) & \textarabic{ث}&& ʿ    & \textarabic{ع}  &     \\
&j    & \textarabic{ج}&& gh   & \textarabic{غ}  &     \\
&H    & \textarabic{ح}&& f    & \textarabic{ف}  &     \\
&kh   & \textarabic{خ}&& (q)  & \textarabic{ق}  &     \\
&d    & \textarabic{د}&& k    & \textarabic{ك}  &     \\
&(dh) & \textarabic{ذ}&& l    & \textarabic{ل}  &     \\
&r    & \textarabic{ر}&& m    & \textarabic{م}  &     \\
&z    & \textarabic{ز}&& n    & \textarabic{ن}  &     \\
&s    & \textarabic{س}&& h    & \textarabic{ه}  &     \\
&sh   & \textarabic{ش}&& w    & \textarabic{و}  &     \\
&S    & \textarabic{ص}&& y    & \textarabic{ي}  &    \\
                 
\end{tabularx}

Phonemes within parenthesis are specific to Standard Arabic and are not part of the Syrian Arabic phonemic system. Their respective letters are however commonly used in Syrian-Standard cognates in written vernacular.\footnote{For the us connected letter forms, see \textit{The~Arabic Writing System} (\url{http://andreasmhallberg.github.io/documents/arabic-letters-and-vowel-markers.tex.pdf})} 

\end{multicols}

\subsection{Vowels}
\needspace{5cm}


\begin{multicols}{2}

  \begin{tabularx}{\linewidth}{X>{\itshape}l>{\itshape}lcX}

    & \upshape Short & \multicolumn{2}{l}{Long} & \\
    & i              & ii                       & \textarabic{ي} & \\
    & u              & uu                       & \textarabic{و} & \\
    & a              & aa                       & \textarabic{ا} & \\
    & e              & ee                       & \textarabic{ي} & \\
    & o              & oo                       & \textarabic{و} & \\

  \end{tabularx}

Only long vowels have an orthographic representation. Vowel diacritics are generally not used in written Arabic vernacular.

\end{multicols}

\subsection{Standard to Syrian phonemic conversion rules}\label{subsec:converston}

\begin{multicols}{2}

  \begin{tabularx}{\linewidth}{>{\itshape}l@{~→~}>{\itshape}lX>{\itshape}l@{~→~}>{\itshape}lr}

  ay                     & ee        &  & bayt     & beet            & \textarabic{بيت}\\
  aw                     & oo        &  & yawm     & yoom            & \textarabic{يوم}\\
  q                      & ʾ         &  & daqiiqa  & daʾiiʾa         & \textarabic{دقيقة}\\
  th                     & t         &  & thalatha & tlaate          & \textarabic{تلاتة}\\
    \multicolumn{1}{c}{} & \llap{(}s) &  & mathalan & masalan         & \textarabic{مثلا}\\
  dh                     & d         &  & dhahab   & dahab           & \textarabic{دهب}\\
    \multicolumn{1}{c}{} & \llap{(}z) &  & idha     & iza             & \textarabic{اذا}\\
  DH                     & D         &  & DHuhr    & Duhr            & \textarabic{ضهر}\\

\end{tabularx}

\columnbreak

These conversion rules represent regular sound relations in Standard--Syrian cognates. Written Syrian Arabic often reflect Standard Arabic orthography. Standard Arabic pronunciation is often retained in formal or literary words (\trans{dimuqraaTiyye}) and in non-Syrian Arabic place names (\trans{al-qaahira} `Cairo'). Standard Arabic \trans{th} and~\trans{dh} are in these context rendered \trans{s} and~\trans{z} respectively.

\end{multicols}


\section{Pronouns}

\subsection{Personal pronouns}

\begin{multicols}{2}

\begin{tabular}{>{\itshape}l>{\itshape}l>{\itshape}lrrr
  }
  \toprule
  \upshape{Indep.} 
  & \multicolumn{2}{l}{Attached}
  &  \multicolumn{2}{r}{\textarabic{متصل}}
  & \textarabic{منفصل}\\
  \midrule

  ana    & -i   & (-ya)  &                    & \textarabic{ـي}                      & \textarabic{انا}    \\
  inti   & -ak  & (-k)   &                    & \textarabic{ـك}                      & \textarabic{انت}    \\
  inta   & -ik  & (-ki)  & \textarabic{(ـكي)} & \textarabic{ـك}                      & \textarabic{انتي}   \\
  huwwe  & -u   & (-h)   & \textarabic{(ـه)}  & \textarabic{ـو\rlap{/\,ـه}}          & \textarabic{هو}     \\
  hiyye  & -a   & (-ha)  &                    & \textarabic{ـها}\rlap{\footnotemark} & \textarabic{هي}     \\
  niHna  & -na  &        &                    & \textarabic{ـنا}                     & \textarabic{نحنا}   \\
  intu   & -kon &        &                    & \textarabic{ـكن}                     & \textarabic{انتو}  \\
  hunnen & -on  & (-hun) & \textarabic{(ـهن)} & \textarabic{ـن}                      & \textarabic{هنن}    \\

\bottomrule

\end{tabular}
\footnotetext{The~\textarabic{ه} in~\textarabic{ـها} is only pronounced if used with a word ending in a vowel.}

\columnbreak

\noindent
The forms in parenthesis are used on words with final vowel: \trans{aʿTaa-ki} `he~gave~you'. Attached pronouns are used for

\begin{tabular}{lll}
   
  prep. compl. & \trans{maʿ-u} &`with-him'\\
   possession& \trans{ktaab-u} &`his book'\\
   direct object& \trans{katabt-u} &`I wrote it'\\
   indirect object& \trans{katabt-l-u} &`I~wrote~for~him'\footnotemark\\
\end{tabular}

The\page{480f} connective~\trans{-l-} `for' for indirect objects may take the form \trans{-el-} to prevent consonant clusters: \trans{katabt‑el‑kun} `I~worte for you~(pl.)'.

\footnotetext{A direct object may then be added with the word \trans{yaa-:} \trans{katabt-l-ak yaa-h} `I~wrote it for you'.}

\end{multicols}

\subsection{Demonstrative pronouns} \page{552}

\begin{tabular}{l>{\itshape}l>{\itshape}lrr}
\toprule

                   & \upshape Proximal & \upshape Distant & \textarabic{بعيد}   & \textarabic{قريب} \\
  \midrule
  Masc.            & haada             & hadaak           & \textarabic{هداك}   & \textarabic{هاد}\\
  Fem.             & haay              & hadiik           & \textarabic{هديك}   & \textarabic{هاي} \\
  Pl.              & hadool            & haadooliik       & \textarabic{هدوليك} & \textarabic{هدول} \\

  \bottomrule
\end{tabular}


\section{Prepositions}

\begin{multicols}{2}

\begin{tabular}{>{\itshape}lrl}


   \ayntab ʿand              & \textarabic{عند} & with, at\\
   \ayntab ʿa-               & \textarabic{عـ}  & to\\
   \ayntab ʿala\footnotemark & \textarabic{على} & on\\
   men                       & \textarabic{من}  & from\\
   Hadd                      & \textarabic{حد}  & next to\\
   fooʾ                      & \textarabic{فوق} & above\\
   taHt                      & \textarabic{تحت} & below\\
   maʿ                       & \textarabic{مع}  & with\\
   wara                      & \textarabic{ورا} & behind\\
   \ayntab ʾuddaam                   & \textarabic{قدام} & in front of\\

  \addlinespace
  \multicolumn{2}{l}{w.noun/w.pronoun}\\
  \addlinespace
   bi-/fii-      & \textarabic{بـ/فيـ}     & in, at\\
   la-/el-       & \textarabic{لـ/الـ}     & for (ownership)\\
   been/beenaat- & \textarabic{بين/بيناتـ} & between, among\\

\end{tabular}

\footnotetext{\trans{ʿala} becomes \trans{ʿalee-} \textarabic{عليـ} when used with a pronominal suffix: \trans{ʿalee-ha} `on~it/her'.}

\columnbreak

  The\page{176ff} latter three prepositions have two separate forms for use with nouns and pronouns respectively:

  \begin{tabular}{lrl}
    \trans{bi-l-bayt}&\textarabic{بالبيت}&`in the house'\\
    \trans{fii-h}&\textarabic{فيه}&`in~it'\\
  \end{tabular}

\end{multicols}

\section{Question words}

\begin{multicols}{2}

\begin{tabular}{>{\itshape}lrl}

    shuu?  & \textarabic{شو؟}    & what?  \\
    miin?  & \textarabic{مين؟}   & who?   \\
    eemta? & \textarabic{ايمتى؟} & when?   \\
    kiif?  & \textarabic{كيف؟}   & how?    \\
    ween?  & \textarabic{وين؟}    & where?  \\
    leesh?  & \textarabic{ليش؟}    & why?  \\

\end{tabular}
  
  Yes/no question formed with rising intonation, or with an elongated clause-final syllable.\page{379} \trans{Ween} and \trans{kiif} take attached pronouns:

  \begin{tabular}{lrl}
    \trans{ween-ak?}  & \textarabic{وينك؟} & `where are you?'\\
    \trans{kiif-ak?}  & \textarabic{كيفك؟} & `how are you?' \\
  \end{tabular}


\end{multicols}

\section{Verbs}

\subsection{Inflection}

\begin{multicols}{2}

  \begin{tabularx}{\linewidth}{l@{}
    >{\itshape}r@{{\itshape-ktib}}>{\itshape}l
    >{\itshape}l@{}
  rr
  }

  \toprule
                 \multicolumn{3}{r}{Non-past}                                                  & \upshape Past & \textarabic{المضارع} & \textarabic{الماضي}\\
  \midrule
  I                                           & i                        &                      & katab-t               & \textarabic{اكتب}  & \textarabic{كتبت}   \\
  you~(ms.)                                   & ti                       &                      & katab-t               & \textarabic{تكتب}  & \textarabic{كتبت}   \\
  you~(fs.)                                   & ti                       & -i                   & katab-ti              & \textarabic{تكتبي} & \textarabic{كتبتي}  \\
  he                                          & yi                       &                      & katab                 & \textarabic{يكتب}  & \textarabic{كتب}     \\
  she                                         & ti                       &                      & katab-it              & \textarabic{تكتب}  & \textarabic{كتبت}   \\
  we                                          & ni                       &                      & katab-na              & \textarabic{نكتب}  & \textarabic{كتبنا}  \\
  you~(pl.)                                   & ti                       & -u                   & katab-tu              & \textarabic{تكتبو} & \textarabic{كتبتو}  \\
  they                                        & yi                       & -u                   & katab-u               & \textarabic{يكتبو} & \textarabic{كتبو}   \\

  \bottomrule
\end{tabularx}

\bigskip


The pronoun is often omitted. The non-past verb is always preceded by one of the following:

\medskip

  \begin{tabular}{lrl}

  \trans{b-}\footnotemark & \textarabic{بـ}    & habitual, generalities\\
  \ayntab\trans{ʿam}    & \textarabic{عم}    & progressive\page{320}\\
  \trans{Ha-/raH}         & \textarabic{حـ/رح} & future\\
  \trans{laazim}         & \textarabic{لازم} & `have to'\\

\end{tabular}
  \footnotetext{For 1pl. the \trans{b-} prefix is partially assimilated and pronounced as \trans{m:} \mbox{\trans{mniktib}}. This is often reflected in Arabic orthography: \textarabic{منكتب}.}

\medskip

  or an auxiliary verb, in which case both the auxiliary and the main verb are inflected for person:

\medskip

  \begin{tabular}{lrl}

      \trans{biddu yiktib}    & \textarabic{بدو يكتب} & `he wants to write'\\
      \trans{byiHibb yiktib}  & \textarabic{بيحب يكتب}& `he likes to write' \\
      \trans{kaan yiktib}     & \textarabic{كان يكتب}& `he was writing'   \\

  \end{tabular}

\end{multicols}

\subsection{\trans{Kaan}} 
\begin{multicols}{2}

  \begin{tabular}{l@{}
    >{\itshape}r@{-}>{\itshape}l
    >{\itshape}l
    @{\hskip 0pt plus 100 pt}rr
    }
  \toprule
   \multicolumn{3}{r}{Non-past}&\multicolumn{1}{l}{Past} & \textarabic{المضارع} & \textarabic{الماضي} \\
  \midrule

    I         & i  & kuun   & kent    & \textarabic{اكون}  & \textarabic{كنت}  \\
    you (ms.) & ti & kuun   & kent    & \textarabic{تكون}  & \textarabic{كنت}  \\
    you (fs.) & ti & kuun-i & kent-i  & \textarabic{تكوني} & \textarabic{كنتي} \\
    he        & yi & kuun   & kaan    & \textarabic{يكون}  & \textarabic{كان}  \\
    she       & ti & kuun   & kaan-it & \textarabic{تكون}  & \textarabic{كانت} \\
    we        & ni & kuun   & ken-na  & \textarabic{نكون}  & \textarabic{كنا}  \\
    you (pl.) & ti & kuun-u & ken-tu  & \textarabic{تكونو} & \textarabic{كنتو} \\
    they      & yi & kuun-u & kaan-u  & \textarabic{يكونو} & \textarabic{كانو} \\

  \bottomrule
\end{tabular}

  \trans{Kaan} is used

  \bigskip

  \begin{enumerate}
    \item[(a)] to place equational (verb-less) clauses in past tense: \trans{kaan aHmad Taaleb} `Ahmad was a student'
    \item[(b)] with verbs to express a past ongoing event: \trans{aHmad kaan (ʿam) yidrus} `Ahmed was writing'
  \end{enumerate}

\end{multicols}

\subsection{Pseudo-verbs}

\begin{multicols}{2}

  \begin{tabularx}{\linewidth}{>{\itshape}XlrlX}

    & bidd‑         &\textarabic{بد}& `want'&\\
    & \ayntab ʿind‑ &\textarabic{عند}& `has'&\\
    &fii           &\textarabic{في}& `there is'&\\

\end{tabularx}

\columnbreak

  The pseudo-verbs are negated as verbs with \trans{maa} but do not follow normal person and tense inflection. \trans{bidd‑} and \trans{ʿund-} are inflected for person with attached pronouns, like nouns (\trans{bidd‑u} `he~wants'; \trans{ʿind-u} `he~has'). \trans{fii} is not inflected for person. The pseudo-verbs are inflected for past with the auxiliary verb \trans{kaan}.

\end{multicols}

\section{Negation}

\begin{multicols}{2}

  \begin{tabularx}{\linewidth}{lrl}

  \trans{maa}   & \textarabic{ما} & verbs \\
  \trans{muu}  & \textarabic{مو} & non-verbal clauses \\
  \trans{laʾʾa} & \textarabic{لاء}  & answering in the negative \\

\end{tabularx}

\columnbreak

\end{multicols}

  \begin{tabularx}{\linewidth}{l>{\itshape}Y>{\itshape}Y}

                & \multicolumn{1}{l}{Verbal}           & \multicolumn{1}{l}{Equational} \\
                & `Ahmed '&\\
  \midrule
    Past        & aHmad \textbf{maa} daras              & aHmad \textbf{maa} kaan Taaleb          \\
                & \hfill\textarabic{أحمد ما درس}       & \hfill\textarabic{أحمد ما كان طالب}  \\
    Present     & aHmad \textbf{maa} byidrus           & aHmad \textbf{muu} Taaleb               \\
                & \hfill\textarabic{أحمد ما بيدرس}     & \hfill\textarabic{أحمد مو طالب} \\
    Future      & aHmad \textbf{maa} ḥa‑/raḥ yidrus    & aHmad \textbf{maa} ḥa‑/raḥ yikuun Taaleb\\
                & \hfill\textarabic{أحمد ما حـ/رح يدرس} & \hfill\textarabic{أحمد ما حـ/رح يكين طالب}\\
  \bottomrule

\end{tabularx}


\pagebreak
\section{Numerals} \page{170f}

\newcommand{\onetwoinfo}{The numerals \textit{one} and \textit{two} are used only for emphasis or contrast, or when ordering in restaurants (\trans{itneen shaay} `two tee'). Otherwise, the lone noun in singular or dual is used.}
\newcommand{\threeteninfo}{The \textit{absolute} form of the numeral is used when it not followed by a noun, hand the \textit{construct} form when it followed by a noun: \trans{tlatt awlaad} `three boys.' Numerals 3--10 have a special construct form with a final~\trans{-t} when used with one of three nouns \trans{ayyaam} 'days', \trans{shuhuur} 'months', and \trans{aalaaf} 'thousands': \trans{khamst ayyam} 'five days'.} 
\newcommand{\teensinfo}{Numerals 11---19 are constructed of the unit number and the ending \trans{‑taʿsh}, with som irregularites in 11, 12, and 15. On the construct form the suffix~\trans{-ar} is added.} 
\newcommand{\cententialinfo}{Cententials are constructed of the construct form of the unit and the ending \trans{‑iin}, with only 20 having an irregular form. In complex numbers, the unit number appear before the centential in its absolute form. The unite and the centential are connected with the conjuction~\trans{u-}: \trans{sabʿa u‑tlaatin}, `thirty‑six'.} 

\begin{tabularx}{\textwidth}{l
  >{\itshape}l>{\itshape}l
  rrrX
  }

  \cmidrule{1-5}
  & \upshape Absolute  & \upshape Construct & \textarabic{مضاف}& \llap{\textarabic{غير مضاف}}\\
  \cmidrule{1-5}

  1  & \rlap{waaHid/wahde}                                                                                               &               &                         & \llap{\textarabic{واحد/وحدة}}  && \multirow[t]{4}{\hsize}{\onetwoinfo}    \\

  2              & \rlap{itneen/tinteen} &               &                       & \llap{\textarabic{اتنين/تنتين}}         & \\
  \\
  \\
  \\
  \\
  3              & tlaate              & tlatt         & \textarabic{تلات}      & \textarabic{تلاتة}          && \multirow{7}{\hsize}{\threeteninfo} \\
  4              & arbaʿa              & arbaʿ         & \textarabic{أربع}     & \textarabic{أربعة}         & \\
  5              & khamse              & khams         & \textarabic{خمس}      & \textarabic{خمسة}          & \\
  6              & sitte               & sitt          & \textarabic{ست}       & \textarabic{ستة}           & \\
  7              & sabʿa               & sabʿ          & \textarabic{سبع}      & \textarabic{سبعة}          & \\
  8              & tmane               & tman          & \textarabic{تمان}     & \textarabic{تمانة}         & \\
  9              & tisʿa               & tisʿ          & \textarabic{تسع}      & \textarabic{تسعة}          & \\
  10             & ʿashara             & ʿashar        & \textarabic{عشر}      & \textarabic{عشرة}          & \\
  \\
  \\
  11             & iddaʿsh             & iddaʿshar     & \textarabic{ادعشر}    & \textarabic{ادعش}          && \multirow[t]{9}{\hsize}{\teensinfo}    \\
  12             & itnaʿsh             & itnaʿshar     & \textarabic{اتنعشر}   & \textarabic{اتنعش}    \\
  13             & tlataʿsh            & tlataʿshar    & \textarabic{تلتعشر}   & \textarabic{تلاتعش}    \\
  14             & arbaʿtaʿsh          & arbaʿtaʿshar  & \textarabic{اربعتعشر} & \textarabic{اربعتعش}    \\
  15             & khamastaʿsh         & khamastaʿshar & \textarabic{خمستعشر}  & \textarabic{خمستعش}    \\
  16             & sittaʿsh            & sittaʿshar    & \textarabic{ستعشر}    & \textarabic{ستعش}    \\
  17             & sabʿataʿsh          & sabʿataʿshar  & \textarabic{سبعتعشر}  & \textarabic{سبعتعش}    \\
  18             & tmantaʿsh           & tmantaʿshar   & \textarabic{بماتعشر}  & \textarabic{بماتعش}    \\
  19             & tisʿataʿsh          & tisʿataʿshar  & \textarabic{تسعتعشر}  & \textarabic{تسعتعش}    \\
  \\
  20             & ʿishriin            &               &                       & \textarabic{عشرين}  &&\multirow[t]{9}{\hsize}{\cententialinfo}   \\
  30             & tlaatiin            &               &                       & \textarabic{تلاتين}     \\
  40             & arbaʿiin            &               &                       & \textarabic{اربعين}     \\
  50             & khamsiin            &               &                       & \textarabic{خمسين}     \\
  60             & sittiin             &               &                       & \textarabic{ستين}     \\
  70             & sabʿiin             &               &                       & \textarabic{سبعين}     \\
  80             & tmaaniin            &               &                       & \textarabic{تمانين}     \\
  90             & tisʿiin             &               &                       & \textarabic{يسعين}     \\
  100            & miyye               & miit          &                       & \textarabic{مية}     \\
  \cmidrule{1-5}

\end{tabularx}
\footnotetext{The form \trans{tinteen} is used in connection with feminine nouns.}

\bigskip





\end{document}
