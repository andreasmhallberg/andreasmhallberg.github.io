\documentclass{article}

\frenchspacing

% TYPOGRAPHY
\usepackage{microtype}
\usepackage{booktabs}
\usepackage{array}
\usepackage{/Users/xhalaa/dotfiles/mylatexstuff/ccbyandreas}
\renewcommand{\ccyear}{2020}

\usepackage{enumitem}
\setlist{nosep}

\usepackage{xcolor}

% LANGUAGES
\usepackage{polyglossia}
\setmainlanguage{arabic}
\setotherlanguage{englis}
\newfontfamily\arabicfont[Script=Arabic,Scale=1.4]{Lateef}


\setlength{\parindent}{0pt}
\setlength{\parskip}{\medskipamount}

% Images
% Metadata

\title{Syrian Arabic grammatical cheat sheet}
\author{Andreas Hallberg}
\date{\today}

% leftward compensation for inital ayn and hamza in tabulars
\newcommand{\ayntab}{\null\hspace{-.2em}}

\begin{document}

\maketitle

% put stuff here

This is a brief grammatical summary of the variety of Arabic spoken in and around Damascus. It is intended for quick references of grammatical forms. For more detailed see Cowell's \textit{A Reference Grammar of Syrian Arabic}, to wh are given in parenthesis.

The transcription system used here is an extended version of the system used in \textit{Alif Baa} and Rydings \textit{Refernce Grammar}. It is extended with the long vowels \textit{oo} and \textit{ee}.
 
\section{Phonetic inventory}

\section{Pronouns}

\begin{tabular}{>{\itshape}l>{\itshape}l>{\itshape}l}
  \upshape{Indep.} & \multicolumn{2}{l}{Attached}\\
  \midrule
  & \upshape Cons. & \upshape Vow.\\
  \cmidrule(rl){2-3}
  أنا  & -i    & -ya\\
  إنْتَ & -ak   & -k\\
  إنْتِ & -ik   & -ki\\
  هُوِ & -u    & -h\\
  هيِ & -a    & -ha\\
  niHna  & -na\\
  hum  & -un   & -hun\\
  hunne  & -kun \\

\end{tabular}

The last form is used on words with final vowel: \textit{aʿTaa-ki} 'he~gave~you~(fs.)'. Attached pronouns are used for

\begin{itemize}
  \item direct objects: \textit{katbt-u} 'I~wrote~it'
  \item indirect objects, with \textit{-l-}, \textit{katabt-l-u} 'I~write~for~him'
  \item possessors: \textit{ktaab-u} 'his~book'
\end{itemize}

\section{Prepositions}

\begin{tabular}{>{\itshape}ll}

  bi-  & in, at  (with nouns)\\
  fii- & in, at  (with pronouns)\\
  \ayntab ʿa-  & to\\
  min  & from\\
  ill-  & for (ownership, with pronouns)\\
  li-  & for (ownership, with nouns)\\
  \ayntab ʿind  & with, at\\
  Hadd & next to\\
  fooʾ & above\\
  taHt & below\\

\end{tabular}

\section{Question words}

\begin{tabular}{>{\itshape}ll}

  shuu & what\\
  miin & who\\
  eemta & when\\
  kiif & how\\
  ween & where\\

\end{tabular}
  
Yes-no question formed with rising intonation or an elongated final syllable. Rethorical yes-no questions may be introduced with \textit{hal}.

\section{Verbs}

\begin{tabular}{l
  >{\itshape}r@{{\itshape-ktib}}>{\itshape}l
  >{\itshape katab}l
  }

  & \multicolumn{2}{l}{Non-past}&\multicolumn{1}{l}{Past}\\

  \midrule

  I          & i  &   & -t\\
  yous (ms.) & ti &   & -t\\
  you (fs.)  & ti & -i & -ti\\
  he         & yi &   & \\
  she        & ti &   & -it\\
  we         & ni &   & -na\\
  you (pl.)  & ti & -u & -tu\\
  they       & yi & -u & -u\\

\end{tabular}

The pronoun may be omitted. The non-past verb is always preceded by one of the following:

\begin{itemize}
\item\textit{b-}, generalities, habitual: \textit{byiktib} 'he~writes'. Note:~1pl.~\textit{\textbf{m}niktib}
\item\textit{ʿam}, progressive: \textit{ʿam yiktib} 'he~is~writing'
\item\textit{Ha-/raH}, future: \textit{raH yiktib} 'he~will~write'
\item\textit{laazim}, \textit{laazim yiktib} 'he~must~write'
\item Aux. verb, eg.
  \begin{itemize}
    \item[] \textit{biddu yiktib} 'he~wants~to~write'
    \item[] \textit{biyHibb yiktib} 'he~likes~to~write'
    \item[] \textit{kaan yiktib} 'he~was~writing'
  \end{itemize}
\end{itemize}

\section{Pseudo-verbs}

\textit{bidd-}~'want' and \textit{ʿind-}~'has' are inflected form person with attached pronouns (as were they nouns), \textit{biddu} 'he~wants'; \textit{ʿindu-} 'he~has', but take the verbal negation \textit{maa} and direct objects.


\section{Standard to Syrian sound conversion rules}


{\itshape
\begin{tabular}{l@{~→~}ll@{~→~}l}

  ay & ee & bayt & beet \\
  aw & oo & yawm & yoom \\
  q  & ʾ  & qaal & ʾaal \\
  th & t*  & thalatha & tlaate \\
  dh & d*  & haadha & haada \\
  DH & D  & DHuhr & Duhr \\

\end{tabular}
}

* For loans fro Standard Arabic \textit{th} and~\textit{dh} becomes \textit{s} and~\textit{z}: \textit{musaqqaf} 'intellectual', \textit{iza} 'if'.

\section{Numbers}

\section{Negation}

\end{document}
